\documentclass[a4paper,12pt]{article}
\usepackage{stilebase}
% \usepackage{float}
% \usepackage{figure}


\title{Appunti di Geometria Affine}
\author{Claudio Afeltra \and Marco Trevisiol}

\begin{document}

\maketitle
% \clearpage


%\begin{abstract}
%	Trattiamo in queste brevissime dispense i fatti 	
%\end{abstract}
%\clearpage

\section{Equivalenza affine}

\begin{definition}[Equivalenza rispetto a $G$]
	Se $G$ è un gruppo di trasformazioni di $\K^n$ e $F_1$ e $F_2$ sono sottoinsiemi di $\K^n$ allora $F_1$ e $F_2$ si dicono equivalenti per 
	$G$ se $\exists g\in G$ tale che $g(F_1)=F_2$.
\end{definition}

In particolare:

\begin{definition}
	$F_1,F_2\in\K^n$ si dicono affinemente equivalenti se $\exists g \in \Aff{\K^n}$ tale che $g(F_1) = F_2 .$
	Analogamente $F_1 , F_2 \in \K^n$ si dicono metricamente equivalenti se $\exists g \in \Isom{\K^n}$ tale che $g(F_1) = F_2 .$
\end{definition}
\begin{remark}
	In generale, data una classe di oggetti $X$ e un gruppo $G$ che agisce su $X$ (per esempio mediante applicazione a sinistra: $gx=g(x)$),
	otteniamo naturalmente una relazione di equivalenza dentro $X$ data da $x_1\sim x_2$ quando esiste $g\in G$ tale che $ x_1 = gx_2$.
	Le classi di equivalenza sono dette orbite dell'azione di $G$ in $X$. Studiamo in questi appunti le classi così ottenute dall'azione
	di particolari applicazioni di uno spazio vettoriale sull'insieme di ipersuperfici affini.
\end{remark}




\begin{definition}[Supporto di un polinomio]
	Dato $g\in \K[x_1,\dots,x_n]$ chiamiamo supporto di $g$ in $\K^n$ l'insieme $V(g)=\{x\in \K^n : g(x)=0\}$.
\end{definition}

\begin{definition}[Polinomi proporzionali]
	Dati $g_1,g_2\in \K[x_1,\dots,x_n]$ diciamo che sono proporzionali $(g_1\sim g_2)$ se $\exists \alpha \in \K^{*}$ tale che $g_1=\alpha g_2$.
\end{definition}
\begin{remark}
	\'E facile verificare che $\sim$ è una relazione di equivalenza
	(discende banalmente dal fatto che $\K^{*}$ è un gruppo rispetto la moltiplicazione).
\end{remark}

\begin{definition}[Ipersuperficie affine]
	Ogni classe di equivalenza di polinomi $[g]$ in $\K[x_1,\dots,x_n]$ rispetto $\sim$ è detta ipersuperficie.
	Inoltre $g(x)=0$ è detta equazione dell'ipersuperficie
\end{definition}

\begin{definition}[Ipersuperfici affinemente equivalenti]
	Date due ipersuperfici affini $[g],[h]$, si dicono affinemente equivalenti se esiste $\Psi\in \Aff{\K^n}$ tale che $[g]=[f\circ\Psi]$.
	In tal caso indicheremo $g\affeq h$
\end{definition}
\begin{remark}
	Useremo anche la notazione $\Psi^{-1}[g]$ per una ipersuperficie equivalente a $[g]$ mediante l'affinità $\Psi \in \Aff{K^n}$.
\end{remark}
\begin{remark}
	Anche in questo caso è facile verificare che la relazione $\affeq$ è di equivalenza, dato che $\Aff{\K^n}$ è un gruppo.
\end{remark}
\begin{remark}
	Notiamo che $[g]\affeq [h]\Rightarrow \text{deg}(g)=\text{deg}(h)$,
	poiché il grado di $g(\Psi(x))$ non può essere più grande di quello di $g(x)$, e il viceversa vale per l'invertibilità di $\Psi$.
\end{remark}

A questo punto passiamo allo studio di alcune ipersuperfici affini tra le più importanti. In particolare tratteremo il caso in cui le ipersuperfici
sono di primo grado e quello in cui sono di grado due (dette quadriche) nei casi particolari in cui $\K=\R$ oppure $\K=\C$.

\begin{proposition}[Grado $1$]
	Tutti i polinomi di primo grado sono affinemente equivalenti;
	ossia esiste un'unica classe di equivalenza per le ipersuperfici di primo grado.
\end{proposition}

\begin{proof}
	Notiamo preliminarmente che un polinomio di primo grado possiamo denotarlo con $g(x)=\tra{A}x+b$ dove $A\in \K^n$, $A\neq 0$ e $b\in \K$.
	Consideriamo ora il polinomio $h(x)=\tra{A'}x+b'$ e cerchiamo $\Psi\in\Aff{K^n}$ tale che $g\circ \Psi = h$, ovvero che
	\[
		\tra{A}(Mx+N)+b=\tra{A'}x+b'
	\]
	dove $M\in \GL{\K^n}$ e $N\in \K^n$. In particolare vogliamo che esistano tali $M$ e $N$.
	Ma è ovvio che esiste $M$ tale che $\tra{A}M=\tra{A'}$ e $N$ tale che $\tra{A}N+b=b'$ dato che $A\neq 0$.
\end{proof}


\section{Quadriche reali e complesse}

Consideriamo ora il caso delle quadriche, ossia le ipersuperfici $[g]$ dove $g\in \K^n[x_1,\dots x_n]$
(qui $\K$ è il campo dei reali o dei complessi) e deg$(g)=2$. Cominciamo con
\begin{remark}
	Se $g$ è una quadrica, allora esistono $A\in M(n,\K)$ $A$ simmetrica e non nulla, $B\in \K^n$, $c\in \K$ tali che
	\[
		g(x) = \tra{x}Ax+2\tra{B}x+c.
	\]
\end{remark}

\begin{remark}\label{oss:isomf}
	Con un'altra osservazione possiamo ridurre ancora la notazione. Consideriamo $Q\in M(n+1,\K)$ associata al polinomio di secondo grado $g$:
	\[
		Q=\left(
			\begin{array}{cc}
			A & B\\
			\tra{B} & c
			\end{array}
		\right)
	\]
	e consideriamo anche $\tilde{x} =\left(\begin{smallmatrix}
x \\
1
\end{smallmatrix}
\right)$. Allora $g(x)=\tra{\tilde{x}}Q\tilde{x}$.
\end{remark}

\begin{definition}[Cono]
	Sia $X\subseteq V$ con $V$ uno spazio vettoriale. Diciamo che $X$ è un cono se $x\in X\Rightarrow \lambda x\in X$ per ogni $\lambda \in \K$.
\end{definition}
\begin{lemma}
	Se $g(x)=\tra{x}Qx$ allora $V(g)$ è un cono.
\end{lemma}
\begin{proof}
	Ovvio per l'omogeneità di $g(x)$.
\end{proof}

Data la \cref{oss:isomf}, possiamo equivalentemente parlare dello studio delle quadriche di $\K^n$ come delle quadriche omogenee in $\K^{n+1}$.
Consideriamo finalmente il cambiamento di forma delle  quadriche via affinità.

\begin{lemma}
	Data una quadrica $g$ a cui è associata la matrice $Q$ e una affinità $\Psi(x)=Mx+N$ a cui è possibile  associare la matrice
	$\tilde M_N = \left(\begin{smallmatrix}
M & N \\ 0 & 1
\end{smallmatrix}
\right)$ allora 
	$g\circ \Psi = \tra{\tilde M_N}Q \tilde M_N$.
\end{lemma}
\begin{proof}
	Basta fare un semplice conto: $g(\Psi(x))=\tra{\Psi(x)}Q \Psi(x) = \tra{x}\tra{\tilde M_N}Q \tilde M_N x$ da cui la tesi.
	Tuttavia se svolgiamo il conto tenendo presente che  $\tilde M_N$ e $Q$ sono matrici $n+1\times n+1$ otteniamo
	\[
		\tra{\tilde M_N}Q \tilde M_N = 
		\left(
			\begin{array}{cc}
			\tra{M} & 0\\
			\tra{N} & 1
			\end{array}
		\right)
		\left(
			\begin{array}{cc}
			A & B\\
			\tra{B} & c
			\end{array}
		\right)
		\left(
			\begin{array}{cc}
			M & N\\
			0 & 1
			\end{array}
		\right) =
		\left(
			\begin{array}{cc}
			\tra{M}AM & \tra{M}AN+\tra{M}B\\
			\tra{(\tra{M}AN+\tra{M}B)}& \tra{N}AN+2\tra{B}N+c
			\end{array}
		\right)
	\]

\end{proof}




\end{document}
