\documentclass[a4paper,12pt]{article}
\usepackage{stilebase}
% \usepackage{float}
% \usepackage{figure}


\title{Appunti di Geometria Affine}
\author{Claudio Afeltra \and Marco Trevisiol}

\begin{document}

\maketitle
% \clearpage


%\begin{abstract}
%	Trattiamo in queste brevissime dispense i fatti 	
%\end{abstract}
%\clearpage

\section{Equivalenza affine}

\begin{definition}
 Se $G$ è un gruppo di trasformazioni di $\K^n$ e $F_1$ e $F_2$ sono sottoinsiemi di $\K^n$ allora $F_1$ e $F_2$ si dicono equivalenti per 
 $G$ se $\exists g\in G$ tale che $g(F_1)=F_2$.
\end{definition}

In particolare:

\begin{definition}
$F_1,F_2\in\K^n$ si dicono affinemente equivalenti se $\exists g \in A\!f\!\!f(\K^n)$ tale che $g(F_1) = F_2 .$
Analogamente $F_1 , F_2 \in \K^n$ si dicono metricamente equivalenti se $\exists g \in I\!som(K^n)$ tale che $g(F_1) = F_2 .$
\end{definition}



\end{document}
