 
 \section{Gruppi di trasformazioni}
 
 Sia $ X $ un  insieme non vuoto. Si definisce $$S(X)=\{f:X\to X\quad biunivoche\}.$$ 
 Allora $(S(X),\circ)$ è un gruppo.
 
 \begin{definition}
 Si dice gruppo di trasformazioni di $X$ ogni sottogruppo di $S(X).$
 \end{definition}
 
 \begin{example}
 $GL(V)$ (dove $V$ è uno spazio vettoriale); $O(V,\phi)$ con $\phi\in PS(V).$
 \end{example}
 
 \subsection{Traslazioni}
  
 \begin{definition}
  Dato uno spazio vettoriale $V$ ed un suo elemento $v$ si dice traslazione di vettore $v$ la funzione
 \begin{eqnarray*}
	\tau_v:V & \to & V \\
	w & \mapsto & w+v
 \end{eqnarray*}
 \end{definition}
 
 \begin{remark}
  $\tau_v$ è lineare se e solo se $v=0$.
 \end{remark}
 
 \begin{proposition}
 Sia $T(V)=\{\tau_v\;|\;v\in V\}$. Allora $(T(V),\circ)$
 è un gruppo abeliano di trasformazioni di $V$, ed è isomorfo a $(V,+)$.
 \end{proposition}
 
 \begin{proof}
La proprietà associativa vale perché l'insieme è formato da funzioni.
 $\tau_0$ è evidentemente l'elemento neutro, e $\forall v\in V\;\; \tau_v\circ\tau_{-v}=\tau_{-v}\circ\tau_v=id=\tau_0$, 
 dunque $\tau_v^{-1}=\tau_{-v}$.
 Inoltre $\forall u,v,w\in V\quad (\tau_v\circ\tau_w)(u)=\tau_v(\tau_w(u))=\tau_v(u+w)=u+w+v=\tau_w(\tau_v(u))=$
 $=(\tau_w\circ\tau_w)(v)=(\tau_w\circ\tau_v)(u)$, dunque  $\tau_v\circ\tau_w(u)=\tau_w\circ\tau_v$,
  pertanto il gruppo è abeliano.
 Per la seconda parte della tesi, la funzione
 \begin{eqnarray*}
	(V,+) & \to & (T(V)) \\
	v & \mapsto & \tau_v
 \end{eqnarray*}
 è un isomorfismo di gruppi, perché è ovviamente biunivoca, e  $\tau_{v+w}=\tau_v\circ\tau_w$.
  \end{proof}
  
	\subsection{Azione di un gruppo}

 In questa parte del corso di Geometria Analitica è interessante studiare come certi gruppi di trasformazioni agiscono su punti o insiemi di punti.
 Per questo introduciamo brevemente dei concetti astratti di algebra legati alla teoria dei gruppi.
 
 \begin{definition}[Azione di un gruppo]
	Sia $G$ un gruppo e $X$ un insieme. Si definisce azione di $G$ su $X$ un qualunque omomorfismo $\psi: G \rightarrow S(X)$.
 \end{definition}
 
 \begin{remark}
	Un'azione $\psi$ di $G$ su $X$ induce la seguente relazione $\sim$:
	\[
		x \sim y \Leftrightarrow \exists g\in G : \psi(g)(x)=y;
	\]
	ed è facile vedere che $\sim$ è d'equivalenza.
 \end{remark}

 \begin{definition}[Orbita]
	Ogni classe $[x]\in X/\!\!\sim$ è detta orbita dell'azione $\psi$.
 \end{definition}
 \begin{definition}[Azione transitiva]
	Un'azione $\psi$ su $X$ si dice transitiva se induce un'unica orbita.
 \end{definition}

 \begin{example}
 $T(V)$ agisce su $V$ in modo transitivo.
 Infatti se $v,w\in V,\quad\tau_{w-v}(v)=w$.
 \end{example}
 
 \begin{definition}[stabilizzatore]
	Si definisce stabilizzatore di $x\in X$ il sottogruppo di $G$ dato da
	\[
		Stab_G(x) = \{g\in G : \psi(g)(x)=x\}.
	\]
 \end{definition}

 
	\subsection{Gruppo delle isometrie}
 
 Sia $(V,\phi)$ uno spazio euclideo. $\phi$ induce una distanza $d$ su $V$. Allora si pone
 $$\Isom{V,d}=\{f:V\to V\;|\;\forall P,Q\in V\quad d(P,Q)=d(f(P),f(Q))\}.$$
 
 \begin{remark} $\Isom{ V,d}$ gode delle seguenti proprietà:
 \begin{itemize}
	\item $O(V,\phi)\subseteq \Isom{ V,d}$;
	\item $T(V)\subseteq \Isom{ V,d}$;
	\item $\forall v\in V,\forall f\in O(V,\phi)\quad\tau_v\circ f\in \Isom{ V,d}$
 \end{itemize}
 \end{remark}
 
 \begin{lemma}
 $\forall v\in V, \forall f\in GL(V)$ (e dunque in particolare per 
 $f\in O(V,\phi)$) $f\circ\tau_v=\tau_{f(v)}\circ f$ (e pertanto in generale $f$ e $\tau_v$ non commutano).
 \end{lemma}
 
 \begin{proof}
 $(f\circ\tau_v)(x)=f(x+v)=f(x)+f(v)=\tau_{f(v)}(f(x))=(\tau_{f(v)}\circ f)(x)$.
  \end{proof}
 
 \begin{proposition}
 $\{\tau_v\circ f\;|\;v\in V,f\in O(V,\phi)\}$ è un gruppo rispetto alla composizione.
 \end{proposition}
 
 \begin{proof}
 Poiché  $(\tau_v\circ f)\circ(\tau_w\circ g)=\tau_v\circ(f\circ\tau_w)\circ g=
 \tau_v\circ\tau_{f(w)}\circ f\circ g$, tale insieme è chiuso rispetto alla composizione.
 Inoltre per questo è vero che $(\tau_v\circ f)\circ(\tau_w\circ g)=id$ se e solo se $\tau_v\circ\tau_{f(w)}\circ f\circ g$,
 la qual cosa è senza dubbio vera se $f(w)=-v$  e  $f\circ g=id$, ossia se $w=-f^{-1}(v)$  e 
 $g=f^{-1}$.
  \end{proof}
 
 \begin{remark}
 Analogamente si dimostra che anche $\{\tau_v\circ f\;|\;v\in V,f\in GL(V)\}$ 
 è un gruppo di trasformazioni.
 \end{remark}
 
 \begin{theorem}
 \label{thm:isometrie}
 $\Isom{ V,\phi}=\{\tau_v\circ f\;|\;v\in V,f\in O(V,\phi)\}$.
 \end{theorem}
 
 \begin{proof}
 Il fatto che il membro destro dell'uguaglianza sia incluso in quello sinistro è ovvio.
 Viceversa, se $f\in \Isom{ V,\phi}$  e  $f(0)=v$, allora se  $\tau_{-v}\circ f,\; g\in \Isom{ V,\phi}$ e
 $g(0)=0$. Dunque, per un teorema visto, $g\in O(V,\phi)$, e pertanto $f=\tau_v\circ g$.
  \end{proof}
 
 \begin{corollary}
$\Isom{ V,\phi}$ è un gruppo di trasformazioni.
 \end{corollary}
 
 
\section{Isometrie di $\R^n$}
 
 Si consideri adesso come caso particolare $\R^n$ dotato del prodotto scalare ordinario.
 Il \cref{thm:isometrie} allora implica che $$\Isom{ \R^n}=\{X\mapsto AX+B|A\in O(n),B\in \R^n\}.$$
 Pertanto se $f\in \Isom{ \R^n}$ allora $$Fix(f)=\{X\in \R^n|AX+B=X\}=\{X\in \R^n|(A-I)X=-B\},$$
 quindi o $Fix(f)$ è vuoto, o è un sottospazio affine di $\R^n$ di giacitura
 $$\{X\in \R^n|(A-I)X=0\}=Fix(A).$$
 
 
\subsection{Simmetrie}
 
 $f\in \Isom{ \R^n}$ si dice simmetria se $f^2=id$.
 
 \begin{proposition}
 Sia $f$ una simmetria di $\R^n$ tale che $f(X)=AX+B$, con $A\in O(n$).
 Allora:
 \begin{enumerate}[label=\bf\Roman*)]
	\item $A^2=I$ (e dunque $A$ è diagonalizzabile) e $f(B)=AB+B=0$;\label{sim:p}
	\item $\frac{B}{2}\in Fix(f)$, e dunque $Fix(f)=Fix(A)+\frac{B}{2}$;
	\item $B$ è ortogonale a $Fix(A)$.
 \end{enumerate}
 \end{proposition}
 
 \begin{proof}
 \begin{enumerate}[label=\bf\Roman*)]
 \item $\forall X\in \R^n,  X=f^2(X)=A(AX+B)+B=A^2X+AB+B$, e dunque
 $A^2=I$ e $f(B)=AB+B=0$.
 \item $f(\frac{B}{2})=\frac{AB}{2}+B=\frac{AB+B}{2}+\frac{B}{2}=\frac{B}{2}$.
 \item Poiché $A^2=I,\quad A$ è diagonalizzabile con autovalori $1$ e $-1$, pertanto
 $R^n=V(1,A)\oplus V(-1,A)$. Per definizione si ha che $Fix(A)=V(1,A)$.
 Vogliamo dimostrare che $Fix(A)^{\bot}=V(-1,A)$.  Se $x\in V(-1,A)$ e $y\in V(1,A)$, allora 
  \[
   \langle x,y\rangle=^txy=^t(-Ax)Ay=-^tx^tAAy=-^txy= -\langle x,y\rangle,
  \]
 pertanto $\langle x,y\rangle=0.$
 Dunque $V(-1,A)\subseteq Fix(A)$. Ma entrambi hanno dimensione
 $n-dim(V(1,A))$, dunque coincidono.
 Ma per il punto \ref{sim:p} $B\in V(-1,A)$.
 \end{enumerate}
 \end{proof}
 
	\subsection{Riflessioni}
 
 $f\in \Isom { \R^n }$ viene detta riflessione se $f^2=id$ e $Fix(f)$ 
 è un iperpiano affine (ossia se ha dimensione $n-1$).
 
 \begin{remark}
 Se $f(X)=AX+B$ è una riflessione, $dim(Fix(f))=n-1$, per cui
 $A\in O(n)$ induce una riflessione lineare rispetto alla giacitura di $Fix(f)$. In particolare $det\,A=-1$.
 \end{remark}
 
 \begin{exercise}
In $\R^2$ siano $r_1$ e $r_2$ due rette passanti per l'origine. Sia $\alpha$
 l'angolo (considerato in senso antiorario) tra $r_1$ e $r_2$. Siano $\rho_1$ e $\rho_2$
 le riflessioni del piano di assi rispettivamente $r_1$ e $r_2$.
 Si dimostri che $\rho_2\circ\rho_1$ è la rotazione antioraria del piano avente come centro l'origine e di angolo 
 $2\alpha$.
 \end{exercise}
 
 \begin{exercise}
 Studiare la composizione di due riflessioni distinte di $\R^n$.
 \end{exercise}
 
 \begin{proof}
Siano $\rho_1(X)=A_1X+B_1$ e $\rho_2(X)=A_2X+B_2$, $Fix(\rho_1)=H_1$ e
 $Fix(\rho_2)=H_2$. 
 Sappiamo che $A_1^2=A_2^2=I$ e $A_1B_1+B_1=A_2B_2+B_2=0$.
 
 \begin{description}
  \item[I caso:] $H_1$ e $H_2$ sono paralleli (ossia hanno la stessa giacitura).
 Allora $Fix(A_1)=Fix(A_2)$. Pertanto $A_1$ e $A_2$ inducono una riflessione rispetto allo stesso piano.
 Dunque $A_1=A_2=A$.
 Ma allora $(\rho_2\circ\rho_1)(X)=A(AX+B_1)+B_2=A^2X+AB_1+B_2=X+(AB_1+B_2)=$
 $=X+(B_2-B_1)$
 Di conseguenza $\rho_2\circ\rho_1$ è una traslazione di $B_2-B_1$, che è il doppio della distanza tra
 $H_1$ e $H_2$.
 \item[II caso: ] $H_1$ e $H_2$ non sono paralleli.
 In tal caso $H_1\cap H_2$ è un sottospazio affine di dimensione $n-2$.
 Ma $H_1\cap H_2\subseteq Fix(\rho_1\circ\rho_2)$.
 Inoltre la giacitura di $Fix(\rho_2\circ\rho_1)$ è
 $Fix(A_2A_1)=V(1,A_2A_1)$. Allora $L=V(1,A_2A_1)^{\bot}=V(-1,A_2A_1)$ è invariante per $A_2A_1$.
 Pertanto tutti i piani ortogonali a $H_1\cap H_2$ (che hanno giacitura $L$) sono invarianti per
 $\rho_2\circ\rho_1$.
 Per l'esercizio precedente su tali piani $\rho_2\circ\rho_1$ è composizione di riflessioni rispetto a rette incidenti,
 dunque è una rotazione.
 \end{description}
 \end{proof}
 
 \begin{exercise}[Forma canonica per le simmetrie]
Dimostrare che se $f\in \Isom{ \R^n}$ è una simmetria e
 $k=dim(Fix(f)),$ allora esiste una base ortonormale $B=\{v_1,\ldots,v_n\}$ di $\R^n$ rispetto alla quale
 $f$ si scrive come
 $$f(X)=\left(
 \begin{array}{cc}
  I_k & 0\\
  0 & I_{n-k}
 \end{array}
 \right)
 X+\alpha v_n.
 $$
 \end{exercise}
 
 \begin{theorem}
 Ogni $f\in \Isom{ \R^n}$ è composizione di al più $n+1$ riflessioni
 \end{theorem}
 
 \begin{proof}
 Se $f(0)=0$, allora $f\in O(n)$, dunque è composizione di al più $n$ 
 riflessioni.
 Se $f(0)=B\ne 0$, sia $H=\{X\in \R^n|\;d(X,0)=d(X,B)\}$.
 $H$ è un iperpiano. Infatti 
 \[
  \quad x\in H\Longleftrightarrow ||x||^2=||x-B||^2\Longleftrightarrow\langle x,x\rangle=
 \langle x-B,x-B\rangle\Longleftrightarrow\langle B,B\rangle-2\langle B,X\rangle=0,
 \]
 equazione che definisce un piano. Inoltre, poiché $\frac{B}{2}\in H$, la giacitura di $H$ è 
 $Span(B)^{\bot}$.
 Sia $\rho_H$ la riflessione rispetto all'iperpiano $H$.
 Allora
 
 \[ \left\{
 \begin{array}{l}
  \rho_H\circ f\in \Isom{ \R^n} \\
  (\rho_H\circ f)(0)=\rho_H(B)=0
 \end{array}
 \right.
 \Longrightarrow \rho_H\circ f\in O(n)\Longrightarrow \rho_H\circ f
 \]
 è composizione di al più $n$ riflessioni, dunque $f=\rho_h\circ(\rho_H\circ f)$
 è composizione di al più $n+1$ riflessioni.
  \end{proof}
 
 \begin{definition}Un' isometria si dice diretta se è composizione di un numero pari di riflessioni, si dice inversa se
 è composizione di un numero dispari di riflessioni. Se $f(X)=AX+B,\quad f$ è diretta se $det\; A=1$,
 è inversa se $det\; A=-1$.
 \end{definition}
 
 \begin{proposition}
 Se $f\in \Isom{ \R^n}$ ha almeno un punto fisso, è composizione di al più $n$ riflessioni.
 \end{proposition}
 
 \begin{proof}
 Sia $f(0)=B\ne 0$ (se $f(0)=0$ il la tesi è manifesta) e sia $Q\in \R^n$
 tale che $f(Q)=Q$. Sia inoltre $H$ definito come nella dimostrazione precedente.
 Allora come prima $\rho_H\circ f\in O(n)$, e poiché $d(Q,0)=d(f(Q),f(0))=d(Q,B),\quad Q\in H$, e pertanto
 $(\rho_h\circ f)(Q)=Q$.
 Perciò $dim(Fix(\rho_H\circ f))\ge 1$, e di conseguenza $\rho_h\circ f$ è composizione di al più
 $n-1$
 riflessioni.
 Dunque $f$ è composizione di al più $n$ riflessioni.
  \end{proof}

	\subsection{Classificazione delle isometrie di $\R^2$}
 
 Alcuni esempi di isometrie di $\R^2$ sono:
 \begin{itemize}
	\item traslazioni;
	\item rotazioni;
	\item riflessioni;
	\item composizioni delle precedenti.
 \end{itemize}
 
 \begin{definition}
 Si chiama glissoriflessione (o riflessione rotatoria) la composizione di una riflessione con una traslazione parallela
 all'asse della riflessione.
 \end{definition}
 
 \begin{remark}
 La glissoriflessione è un'isometria inversa senza punti fissi.
 \end{remark}
 
 \begin{theorem}[di classificazione delle isometrie piane]
 
Ogni isometria di $\R^2$ è di uno dei seguenti tipi:
 \begin{enumerate}[label=\bf\Roman*)]
  \item traslazione (isometria diretta senza punti fissi);
  \item rotazione (isometria diretta con punti fissi);
  \item riflessione (isometria inversa con punti fissi);
  \item glissoriflessione (isometria inversa senza punti fissi).
 \end{enumerate} 
 \end{theorem}
 \begin{proof}
 Sia $f\in \Isom{ \R^2}$. Allora $f$ è composizione di $k$ riflessioni, con
 $k\le 3$.
 Se $k=0,\; f$ è l'identità.
 Se $k=1,\; f$ è una riflessione.
 Se $k=2$, come abbiamo già visto $f$ è una traslazione od una rotazione.
 Se $k=3,\; f=\rho_1\circ\rho_2\circ\rho_3$. Sia $r_i=Fix(\rho_i)$ (con $i=1,2,3$).
 $\rho_1\circ\rho_2$ può essere una traslazione od una rotazione.
 \begin{description}
 \item[I caso:] $\rho_1\circ\rho_2=\tau_v$ è una traslazione.
 Allora sia $v=v_1+v_2$, con $v_1//r_3$ e $v_2\bot r_3$.
 Quindi $f=\tau_v\circ\rho_3=\tau_{v_1}\circ(\tau_{v_2}\circ\rho_3)$. Ma $\tau_{v_2}\circ\rho_3$
 è una riflessione
 $\rho'_3$ rispetto ad un asse $r'_3$ parallelo a $r_3$.
 Dunque $f=\tau_{v_1}\circ\rho'_3$, che è una glissoriflessione se $v_1\ne 0$, è una riflessione se 
 $v_1=0$.
 \item[II caso:] $\rho_1\circ\rho_2$ è una rotazione rispetto ad un punto $P$ di angolo $\alpha$.
 Vanno distinti due ulteriori sottocasi:
 \begin{itemize}
  \item $P\notin r_3$. In tal caso la rotazione $R=\rho_1\circ\rho_2$ è composizione di due riflessioni
  $\rho'_1$
  e $\rho'_2$, rispetto a rette $r'_1$ e $r'_2$ incidenti in $P$,
  che formano un angolo di $\frac{\alpha}{2}$ e con $r'_2$ parallela a $r'_3$.
  Allora $f=\rho_1\circ\rho_2\circ\rho_3=\rho'_1\circ\rho'_2\circ\rho_3$. Ma
  $\rho'_2\circ\rho_3$
  è una traslazione, dunque $f$ è una glissoriflessione.
  \item $P\in r_3$ (ossia $r_1,\; r_2$ e $r_3$ sono incidenti). Allora si può scrivere
  $\rho_1\circ\rho_2$ come $\rho'_1\circ\rho_3$, dove $\rho'_1$ è la riflessione rispetto ad un
  opportuno asse passante per $P$.
  Allora $$f=\rho_1\circ\rho_2\circ\rho_3=\rho'_1\circ\rho_3\circ\rho_3=\rho'_1,$$ e dunque $f$ è una rotazione. 
 \end{itemize}
 \end{description}
  \end{proof}
 \begin{remark}
 $\rho_1\circ\rho_2\circ\rho_3$ è una riflessione se e solo se i tre assi di riflessione sono paralleli
 o incidenti.
 \end{remark}
 
  \subsection{Classificazione delle isometrie di $\R^3$}
  
 Alcuni esempi di isometrie di $\R^3$ sono:
 \begin{itemize}
  \item traslazioni (luogo dei punti fissi: $\emptyset$);
  \item riflessioni (luogo dei punti fissi: un piano);
  \item rotazioni (luogo dei punti fissi: una retta).
 \end{itemize}
 \begin{remark}
 La composizione di tre riflessioni rispetto a tre piani distinti incidenti in un punto (ma non in una retta)
 è la simmetria rispetto a quel punto.
 In generale due isometrie elementari (traslazioni, riflessioni, rotazioni) non commutano, fuorché in alcuni casi particolari,
 cui si danno dei nomi:
 \begin{enumerate}[label=\bf\Roman*)]
  \item  si chiama glissorotazione la composizione di una riflessione e di una traslazione parallela al piano di riflessione;
  \item si chiama riflessione rotatoria la composizione di una riflessione e di una rotazione intorno ad una retta ortogonale
  al piano di riflessione (un caso particolare sono le simmetrie centrali);
  \item si chiama avvitamento la composizione di una rotazione e di una traslazione parallela alla retta intorno a cui
  avviene la rotazione.
 \end{enumerate}
 \end{remark}
 \begin{theorem}
 Tutte le isometrie di $\R^3$ sono traslazioni, riflessioni, rotazioni, glissorotazioni, riflessioni rotatorie
 o avvitamenti.
 \end{theorem}
 
  \section{Il gruppo di trasformazioni $A(V)$}
 Dato uno spazio vettoriale $V$ si definisce $A(V)=\{\tau_v\circ f\;|\; v\in V,\; f\in GL(V)\}$.
 È stato già visto che $A(V)$ è un gruppo di trasformazioni, e che coincide con il sottogruppo di $S(V)$
 generato da $T(V)$ e da $GL(V)$.
 
 \vspace{0.5cm}
 
 \begin{definition}
 Dati un gruppo di trasformazioni $G$ di un insieme $X$ e un elemento $x$ di
 $X$ si chiama stabilizzatore di $x$ l'insieme $St_x(G)=\{g\in G\;|\; g(x)=x\}$.
 \end{definition}
 
 \begin{proposition}
 \begin{enumerate}[label=\Roman*)]
	 \item Se $|V|>2$ allora $St_v(GL(V))=GL(V)$ se e solo se $v=0$.
	 \item $\forall v\in V\quad St_v(A(G))\cong GL(V)$.
	\end{enumerate} 
 \end{proposition}
 
 \begin{proof}
 I) Ovviamente $St_0(GL(V))=GL(V).$
 D'altra parte se $v\ne 0$ allora è banale trovare una funzione $f\in GL(V)$ tale che $f(v)\ne v.$
  
 
 II) Innanzitutto $\forall v,w\in V\;\; St_v(A(V))$ e $St_w(A(V))$ sono isomorfi.
 Infatti se
 $u=v-w$ (sicché $\tau_u(w)=v)$
 \begin{eqnarray*}
	St_v(A(V)) &\to & St_w(A(V)) \\
	f &\mapsto & \tau_{-u}\circ f\circ\tau_u
 \end{eqnarray*}
 
 è un isomorfismo.
 Inoltre $\tau_v\circ f\in St_0(A(V))$ se e solo se $(\tau_v\circ f)(0)=0,$ cioè se e solo se $v=0$,
 dunque $St_0(A(V))=GL(V)$.
 Pertanto $$\forall v\in V\quad St_v(A(V))\cong St_0(A(V))=GL(V).$$
 
 
 
 Informalmente si può dire che in $V$ c'è un punto privilegiato, l'origine, mentre in $A(V)$
 tutti i punti sono equivalenti.
 Rileggendo la dimostrazione si può notare che ad ogni coppia di vettori $v,w\in V$ si è associata una traslazione
 $\tau_{v-w}$, e le traslazioni costituiscono un gruppo isomorfo a $(V,+)$.
 Grazie alla definizione di spazio affine si distingueranno gli elementi di $V$ pensati come punti o come traslazioni.
 \end{proof}
 
	\section{Spazi affini}
 
 Dato uno spazio vettoriale $V$, un insieme non vuoto $A$ si dice spazio affine su $V$
 se esiste una funzione $F:A\times A\to V$ che associa ad ogni coppia di punti $P,Q\in V$ un vettore di $V$,
 denotato $\overrightarrow{PQ}$, in modo da verificare le seguenti condizioni:
 \begin{enumerate}[label=\bf\Roman*)]
	\item $\forall P\in A,\forall v\in V\;\;\exists!\; Q\in A$ tale che $\overrightarrow{PQ}=v$;
	\item $\forall P,Q,R\in A\quad\overrightarrow{PQ}+\overrightarrow{QR}=\overrightarrow{PR}\;\;$
		(relazione di Chasles).
 \end{enumerate}
 
 \begin{remark}
 Dalla relazione di Chasles discende che: 
 \begin{enumerate}[label=\bf\alph*)]
	\item $\forall P\in A\quad\overrightarrow{PP}=0\quad$ (prendendo $P=Q=R);$
	\item $\forall P,Q\in A\quad\overrightarrow{PQ}=-\overrightarrow{QP}\quad$ (prendendo $P=R).$
 \end{enumerate}
 \end{remark}
 
 \begin{example}
	\begin{enumerate}[label=\bf\Roman*)]
 \item $A=V$,
 \begin{eqnarray*}
	F: V\times V & \to & V \\
	(P,Q) & \mapsto & \vet{PQ}\stackrel{def}{=}Q-P
 \end{eqnarray*}
 \item Data $f\in Hom(V,W)$ e dato $b\in W$ siano $A=f^{-1}(b)$ e $V=f^{-1}(0)=ker(f)$.
 Allora $A$ è uno spazio affine su $V$ tramite
 \begin{eqnarray*}
  F: A\times A & \to & V \\
  (a_1,a_2) & \mapsto & \vet{a_1a_2}\stackrel{def}{=}a_2-a_1
 \end{eqnarray*}
 \end{enumerate}
 \end{example}
 
	\subsection{Traslazioni}
 
 Dalla definizione, fissato $v\in V,\;\exists !Q\in A$ tale che $\vet{PQ}=v$.
 Si definisce allora traslazione ogni funzione $\tau_v:A\to A$ tale che $\tau_v(P)=Q,$ dove $\vet{PQ}=v$.
 Inoltre si usa la definition $P+v=\tau_v(P)$.
 Con tale definition si ha che:
 \begin{itemize}
	\item $\vet{P(P+v)}= {v}$;
	\item $P+\vet{PQ}=Q$.
 \end{itemize}
 
 \begin{lemma}
 $\forall P\in A,\; \forall v_1,v_2\in V$ si ha che $(P+v_1)+v_2=P+(v_1+v_2).$
 \end{lemma}
 
 \begin{proof}
 Siano $P_1=P+v_1$ e $P_2=P+v_2$. Allora $v_1=\vet{PP_1}$ e
 $v_2=\vet{PP_2}.$.
 Perciò si ha che $P+(v_1+v_2)=P+(\vet{PP_1}+\vet{PP_2})=P+\vet{PP_2}=P_2$.
 \end{proof}
 
	\subsection{Combinazioni affini}
 
 Fissato $P\in A$, si definisce
 \begin{eqnarray*}
  F_P: A & \to & V \\
  Q & \mapsto & \vet{PQ}
 \end{eqnarray*}
 
 Dagli assiomi segue che $F_P$ è biunivoca e che $F_P(P)=\vet{PP}=0$. Dunque $F_P$ trasforma
 $P$ nell'origine di $V$.
 Dunque si vorrebbe trovare una definizione di ``combinazione affine'' di punti di $A$ che corrisponda alla combinazione
 lineare di vettori. Siano $P_1,P_2,\ldots,P_k\in A$. Per ogni $P\in A\; F_P:$ trasforma
 $P_i$ in $\vet{PP_i}$.
 Dati $t_1,\ldots,t_k\in \K$ esiste $\sum_{i=1}^kt_i\vet{PP_i}\in V$, e dunque anche
 $F_P^{-1}(\sum_{i=1}^kt_i\vet{PP_i})\in A$.
 Affinché il risultato sia indipendente da $P,$ deve valere che
 $$\forall P,Q\in A\quad P+\sum_{i=1}^kt_i\vet{PP_i}=Q+\sum_{i=1}^kt_i\vet{QP_i},$$
 il che accade se
 $$P+\sum_{i=1}^kt_i\vet{PP_i}=P+\sum_{i=1}^kt_i(\vet{PQ}+\vet{QP_i})=P+\left(\sum_{i=1}^kt_i\right)\vet{PQ}+\sum_{i=1}^kt_i\vet{QP_i}$$
 è uguale a
 $$Q+\sum_{i=1}^kt_i\vet{QP_i}=P+\vet{PQ}+\sum_{i=1}^kt_i\vet{QP_i},$$
 ossia se e solo se $\sum_{i=1}^kt_i=1$.
 Da ciò discende la seguente definizione.
 
 \begin{definition}
 Dati $P_1,\ldots,P_k\in A$ e $t_1,\ldots,t_k\in \K$ con $\sum_{i=1}^kt_i=1$,
 si chiama combinazione affine di $P_1,\ldots,P_k$, dato un qualsiasi $P\in A$, 
 $F_P^{-1}(\sum_{i=1}^kt_i\vet{PP_i})$.
 Si definisce inoltre $comb_a(P_1,\ldots,P_n)=\{\text{combinazioni affini di }\K\}.$
 \end{definition}
 
 \begin{example}
 Se $A=\K^n$ e $P_1,P_2$ sono punti distinti, allora
 \begin{eqnarray*}
	comb_a(P_1,P_2)=\{t_1P_1+t_2P_2\;|\;t_1+t_2=1\}=\{tP_1+(1-t)P_2\;|\; t\in \K\}=\\
	=\{P_1+(1-t)(P_2-P_1)\;|\; t\in \K\},
 \end{eqnarray*}
 \end{example}

 che è la retta passante per $P_1$ e $P_2.$ Analogamente la combinazione affine di tre punti non allineati
 è il piano passante per essi.
 
	\subsection{Sottospazi affini}
 
 Un sottoinsieme $H$ di $A$ si dice sottospazio affine se è chiuso per combinazioni affini.
 
 \vspace{0.5cm}
 
 \begin{remark}
 L'intersezione di sottospazi affini è un sottospazio affine.
 \end{remark}
 
 \vspace{0.5cm}
 
 \begin{example}
 Sia $P\in A$, e sia $W$ un sottospazio di $V.$ Allora
 $$F^{-1}_{P_0}(W)=\{P\in A\;|\}\vet{P_0P}\in W=\{P\in V\;|\;P=P_0+w, w\in W\}.$$
 Dunque $F^{-1}_{P_0}(P_0+W)=W$.
 \end{example}
 
 \vspace{0.5cm}
 
 \begin{remark}
 $P_0+W=\{\tau_w(P_0)\;|\;w\in W\}$.
 \end{remark}
 
 \vspace{0.5cm}
 
 \begin{proposition}
 $P_0+W$ è chiuso per combinazioni affini.
 \end{proposition}
 
 \begin{proof}
 Siano $P_0+w_1,\dots,P_0+w_n\in P_0+W$ e $t_1+\ldots+t_n=1$. Allora
 
 \[
	\sum_{i=1}^nt_i(P_0+w_i)=P_0+\sum_{i=1}^nt_i\vet{P_0(P_0+w_i)}=P_0+\sum_{i=1}^nt_iw_i\in P_0+W.
 \]
 \end{proof}
 
 \subsection{Giacitura}
 
 \begin{proposition}
 Sia $H$ un sottospazio affine di $A$. Allora esiste un unico sottospazio vettoriale
 $W_H$ di $V$,  detto giacitura di $H$, tale che $\forall P\in H\quad H=P_0+W_H$.
 \end{proposition}
 
 \begin{proof}
 Dato $P_0\in H$, si cerca $W_H$ tale che verifichi la tesi.
 Poiché $P_0+W_H=F^{-1}_{P_0}(W_H)$, necessariamente
 \[
	W_H=F_{P_0}(H)=\{w\in V\;|\; P_0+w\in H\}.
 \]
 Bisogna verificare che $W_H$ sia un sottospazio di $V.$
 Dati $w_1,w_2\in W_H$ e $\alpha_1,\alpha_2\in \K,$ si ha che $P_0+w_1\in H$ e che
 $P_0+w_2\in H$. Ma allora
 \begin{eqnarray*}
	P_0+\alpha_1w_1+\alpha_2w_2=P_0+\alpha_1\vet{P_0(P_0+w_1)}+\alpha_2\vet{P_0(P_0+w_2)}=\\
	=P_0+(1-\alpha_1-\alpha_2)\vet{P_0P_0}+\alpha_1\vet{P_0(P_0+w_1)}+\alpha_2\vet{P_0(P_0+w_2)}=\\
	=F^{-1}_{P_0}((1-\alpha_1-\alpha_2)\vet{P_0P_0}+\alpha_1\vet{P_0(P_0+w_1)}+\alpha_2\vet{P_0(P_0+w_2)}\in H
 \end{eqnarray*}

 in quanto combinazione affine di $P_0,P_0+w_1$ e $P_0+w_2\in W_H$.
 Quanto all'unicità, bisogna verificare che il ragionamento fatto non dipenda dall'elemento di $H$ scelto, ossia che dati
 $P_1,P_2\in H$ valga che $F_{P_1}(H)=F_{P_2}(H)$.
 Ma $F_{P_1}(H)=\{\vet{P_1Q}\;|\; Q\in H\},$ e se $\vet{P_1Q}\in F_{P_1}(H)$ si ha che
 $$\vet{P_1Q}=\vet{P_1P_2}+\vet{P_2Q}=-\vet{P_2P_1}+\vet{P_2Q}\in F_{P_2}(H),$$
 pertanto $F_{P_1}(H)\in F_{P_2}(H)$, ed analogamente si dimostra l'inclusione inversa.
 \end{proof}
 
 \begin{definition}
 Se $H\subseteq A$ è sottospazio affine, si pone $dim\;H=dim\;W_{H}$.
 \begin{itemize}
	\item $H$ è detto retta se $dim\;H = 1$,
	\item $H$ è detto piano se $dim\;H = 2$,
	\item $H$ è detto iperpiano se $dim\;H = dim\;A-1$.
 \end{itemize}
 \end{definition}
 
 \begin{remark}
 Se $H, L \subseteq A \text{ tali che } H\cap L \neq \emptyset$ sono sottospazi affini, allora
 $W_{H\cap L}=W_H\cap W_L$.
 \end{remark}
 
 \begin{definition}
 Due sottospazi affini si dicono:
 \begin{description}
	\item[\textbullet\ incidenti] se $H\cap L \neq 0$,
	\item[\textbullet\ paralleli] se $W_H\subseteq W_L\ \vee\ W_L\subseteq W_H$.
 \end{description}
 \end{definition}
 
 \begin{remark}
 il parallelismo in generale non è una relazione transitiva, quindi neanche d'equivalenza.
 \end{remark}

 \begin{proposition}
 Dati $P_0,\dots,P_k$,
 \[
	comb_a(P_0,\dots,P_k)= P_0+Span(\vet{P_0P_1},\dots,\vet{P_0P_k}).
 \]
 \end{proposition}
 
 \begin{proof}
 Per definizione di combinazione affine vale $\subseteq$. Per il $\supseteq$ si ha
 \[
	P_0+t_1\vet{P_0P_1}+\dots+t_k\vet{P_OP_k} 
		=P_0+\left(1-\sum_{i=1}^{k}t_i\right)+t_1\vet{P_0P_1}+\dots+t_k\vet{P_OP_k},
 \]
 che sta in $comb_a(P_0,\dots,P_k)$, $\forall t_1,\dots,t_k\in \K.$
 \end{proof}
 
 Più in generale, ponendo $comb_a(X)$ l'insieme di combinazioni affini di tutti i possibili sottoinsiemi finiti di $X$
 si ottiene l'analogo della proposizione appena dimostrata.
 
 \begin{definition}
	poniamo $H+L = comb_a(H\cup L)$.
 \end{definition}
 
 \begin{proposition}[Giacitura di $H+L$] Se $H,L\subseteq A$ sono sottospazi affini, allora
 $\forall P\in H$ e $\forall Q \in L$ si ha che
 \[
	W_{H+L}=W_H+W_L+Span(\vet{PQ}).
 \]
 \end{proposition}

 \begin{proof}
  $H\subseteq H+L \Rightarrow W_H \subseteq W_{H+L}$, $W_L \subseteq W_{H+L}$ ; inoltre si ha che
 $comb_a(P,Q)\subseteq H+L \Rightarrow W_{\vet{PQ}}\subseteq W_{H+L}$; quindi vale l'inclusione
 \[
	W_{H+L}\supseteq W_H+W_L+Span(\vet{PQ}).
 \]
 Sia ora $S=P+W_H+W_L+Span(\vet{PQ})$. Si ha che $S\supseteq P+W_H = H$; e, poiché $Q= P+ \vet{PQ}\in S$,
 $L=Q+W_L \subseteq Q+W_H+W_L+Span(\vet{PQ})=S$. Ciò significa che $S\supseteq H+L$ per definizione di combinazione
 affine.
 \end{proof}
 
 \begin{lemma}
  $H,L\subseteq A$ sottospazi affini, allora vale che
 \[
	H\cap L = \emptyset\quad \Leftrightarrow\quad \forall P\in H, Q \in L,\ \vet{PQ}\notin W_H+W_L.
 \]
 \end{lemma}
 
 \begin{proof}
 Se per assurdo esistessero $P\in H \text{e} Q\in L \text{ tali che } \vet{PQ}=w_1+w_2$ in modo che
 $w_1\in W_H, w_2\in W_L$, si avrebbe che $H \ni P+w_1 = P+(\vet{PQ}-w_2) = Q - w_2 \in L.$
 
 Viceversa, se $\exists R \in H \cap L$, presi $P\in H, Q \in L$ si ha che $H=P+W_H,L=Q+W_L$ quindi
 $\vet{PQ}=\vet{PR}+\vet{RQ}\in W_H+W_L$.
 \end{proof}
 
 \begin{proposition}[Formula di Gra\ss mann affine]
Dati $H,L\subseteq A$ sottospazi affini si ha che
 \begin{enumerate}[label=\bf\Roman*)]
	\item $H\cap L \neq \emptyset, \ dim\;(H+L)=dim\;H+dim\;L-dim\;(H\cap L)+1$;\label{GrassAffnote}
	\item $H\cap L = \emptyset, \ dim\;{H+L}=dim\;H+dim\;L+1$.\label{GrassAffemp}
 \end{enumerate}
 \end{proposition}
 
 \begin{proof}
  $dim\;(H+L)= dim\;W_{H+L}$, $W_{H+L}=W_H+W_L+Span(\vet{PQ}),$ quindi si ha che:
 \begin{description}
	\item[\ref{GrassAffnote}] per il lemma otteniamo che $\vet{PQ}\in W_H+W_L$ cioè $W_{H+L}=W_L+W_H$
		da cui la tesi discende dalla formula di Gra\ss mann vettoriale;
	\item[\ref{GrassAffemp}] in questo caso $\vet{PQ}\notin W_H+W_L$ allora $dim\;W_{H+L}=dim\;(W_L+W_H)+1$
		e la tesi si ha ancora per la formula di Gra\ss mann vettoriale.
 \end{description}
 \end{proof}

 \begin{definition}
  $H,L$ si dicono sghembi se  $H\cap L = \emptyset$ e $W_H \cap W_L = \{0\}$.
 \end{definition}
 
	\subsection{Riferimenti affini}
 
 \begin{definition}
 Dati i punti $P_0,\dots, P_k\in A,$ essi si dicono affinemente indipendenti se si ha
 $dim\;comb_a(P_0,\dots,P_k)=k$.
 \end{definition}
 
 \begin{definition}
 Sia $n=dim\;A$. Ogni $n+1$-upla di punti $\{P_0,\dots, P_n\}$
 che siano affinemente indipendenti.
 \end{definition}
 
 \begin{remark}[passaggio da un riferimento affine ad base di vettori]
 Se $\{P_0,\dots, P_n\}$ è un riferimento affine, allora
 $\{\vet{P_0P_1},\dots,\vet{P_0P_n}\}$ è una base di $V$. Viceversa, se $\{v_1,\dots,v_n\}$ è una base di
 $V$ e $P\in A$ , allora $\{P,P+v_1,\dots,P+v_n\}$ è un riferimento affine. In particolare se
 $A=V=\K^n$, $\{0,e_1,\dots,e_n\}$ è detto riferimento affine standard.
 \end{remark}
 
 \begin{proposition}
 Se $R=\{P_0,\dots,P_k\}$ è un riferimento affine di $A$, allora
 \[
	\forall P\in A, \exists ! a_0,\dots,a_n\in \K :\ P=a_0P_0+\dots+a_nP_n.
 \]
 In particolare, la $n+1$-upla $a_0,\dots,a_n$ sono detti coefficienti affini di $P$
 rispetto a $R$.
 \end{proposition}
 
 \begin{proof}
 Per ipotesi, $A=comb_a(P_0,\dots,P_n)$ quindi esistono sicuramente dei coefficienti per ogni
 $P\in A$. Per mostrare che sono unici, basta spostarsi nello spazio vettoriale
 $V$ con il procedimento mostrato nella nota precedente.
 \end{proof}

	\section{Trasformazioni affini}
	
 Siano $A$ uno spazio affine su $V$ e $B$ uno spazio affine su $W,$ dove $V$ e
 $W$ sono spazi vettoriali sul campo $\K.$
 
 \begin{definition}
 Si dice trasformazione affine una funzione $f:A\to B$ che conserva le combinazioni affini.
 \end{definition}
 
 \begin{definition}
 Una trasformazione affine biunivoca si dice isomorfismo affine.
 \end{definition}
 
 \begin{definition}
 Un isomorfismo affine da uno spazio affine in sé si dice affinità. Si definisce inoltre
 $\Aff{A}=\{f:A\to A\;|\; f\; \text{ è un'affinità}\}.$
 \end{definition}
 
 \begin{example}
 \begin{enumerate}[label=\bf\Roman*)]
  \item Le traslazioni sono affinità.
  \item Se $R$ è un riferimento affine, la funzione
   \begin{eqnarray*}
  [\quad]_R:A & \to & \K^n \\
  P & \mapsto & [P]_R
 \end{eqnarray*}
  (dove $[P]_R$ sono le coordinate affini di $P$ rispetto a $R)$ è un isomorfismo affine;
 esso manda i punti di $R$ in $0,e_1,\ldots,e_n.$
 \end{enumerate}
 \end{example}
 
 \begin{proposition}
 Se $V$ è uno spazio vettoriale, $f:V\to V$ è un'affinità e $f(0)=0,$ allora
 $f$ è lineare.
 \end{proposition}
 \begin{proof}
 $\forall v_1,v_2\in V\quad\forall t_1,t_2\in \K$ vale che:
 
 \begin{eqnarray*}
	& f(t_1v_1+t_2v_2)=f((1-t_1-t_2)\cdotp 0+t_1v_1+t_2v_2=\\
	& =(1-t_1-t_2)f(0)+t_1f(v_1)+t_2f(v_2)=t_1f(v_1)+t_2f(v_2).
 \end{eqnarray*}
 \end{proof}
 
 \begin{proposition}
 Se $f\in \Aff{v}$ allora esistono e sono unici $v\in V$ e $g\in GL(V)$ tali che
 $f=\tau_v\circ g.$
 \end{proposition}
 
 \begin{proof}
 Sia $v=f(0).$ Allora $g=\tau_{-v}\circ f\in \Aff{V}$ e $g(0)=0,$
 quindi per la proposizione precedente $g\in GL(V).$
 Da tale proposizione segue che 
 \[
	\Aff{V}\{\tau_v\circ f\;|\;v\in V,f\in GL(V)\}.
 \]
 Di conseguenza $\Aff{V}$ è il gruppo generato da $T(V)$ e da $GL(V).$
 In particolare 
 \[
	\Aff{\K^n}=\{X\mapsto AX+B \; |\; A\in GL(n,\K),B\in \K^n\}.
 \]
 \end{proof}
 
\begin{proposition}
Dato uno spazio $A$ affine su $V$ e dati i riferimenti affini $L=(P_0,P_1,\ldots,P_n)$
 e $M=(Q_0,Q_1,\ldots,Q_n)$ esiste un'unica affinità $f\in \Aff{A}$ tale che
 $\forall i=0,\ldots,n\ f(P_i)=Q_i.$
 \end{proposition}
 
 \begin{proof}
 Consideriamo le trasformazioni invertibili $F_{P_0}:A\rightarrow V$ e $F_{Q_0}:A\rightarrow V$,
 allora si ha che $F_{P_0}(L)=(0,\vet{P_0P_1},\dots,\vet{P_0P_n})=(0,B)$ dove $B\in V$ è un vettore
 di dimensione $n$; similmente $F_{Q_0}(M)=(0,C).$
 
 Ora, $B,C$, per motivi dimensionali, sono basi dello spazio $V$, pertanto esiste un unico endomorfismo lineare
 $g$ che mappa $B$ in $C$. Infine, sia $\tau:V\rightarrow V$ la traslazione del vettore
 $\vet{P_0Q_0}$. Mostriamo che la mappa $f:A\rightarrow A$ definita da
 \[
	f = F_{P_0}^{-1} \circ \tau \circ g \circ F_{P_0}
 \]
 soddisfa le condizioni richieste:
 \begin{eqnarray*}
	&&f(P_0)=F_{P_0}^{-1} \circ \tau \circ g (0) = F_{P_0}^{-1} \circ \tau (0) = F_{P_0}^{-1} (\vet{P_0Q_0}) = Q_0\\
	&&f(P_i)=F_{P_0}^{-1} \circ \tau \circ g (\vet{P_0P_i}) = F_{P_0}^{-1} \circ \tau (\vet{Q_0Q_i}) = F_{P_0}^{-1} (\vet{P_0Q_i}) = Q_i)
 \end{eqnarray*}
 $\forall i=1,\dots,n.$
 Mostriamo che conserva anche le combinazioni affini, in particolare basta mostrare che conserva quelle di punti in $L$:
 \begin{eqnarray*}
	f\left(\sum_{i=0}^{n}t_iP_i\right)&=&F_{P_0}^{-1} \circ \tau \circ g \left(\sum_{i=0}^{n}t_i\vet{P_0P_i}\right)\\
	&=&F_{P_0}^{-1} \circ \tau \left( \sum_{i=0}^{n}t_i\vet{Q_0Q_i}\right) \\
	&=&F_{P_0}^{-1} \left(\sum_{i=0}^{n}t_i\vet{P_0Q_i}\right)\\
	&=&\sum_{i=0}^{n}t_iQ_i.
 \end{eqnarray*}
 
 D'altra parte, si verifica facilmente che se esiste $f$ siffatta, allora l'applicazione $j:V\rightarrow V$ 
 definita
 \[
	j = \tau^{-1} \circ F_{P_0} \circ f \circ F_{P_0}
 \]
 manda  $\vet{P_0P_i}$ in $\vet{Q_0Q_i}\ \forall i=1,\dots,n$
 e che è lineare (la verifica è del tutto analoga ai passaggi appena descritti). Da ciò segue quindi che  $g=j$ 
 cioè l'unicità.
 \end{proof}
 
	\subsection{Gruppo delle affinità}
 
 \begin{definition}[gruppo delle affinità]
 consideriamo il seguente insieme
 \[
	\Aff{\K^n}=\{x\mapsto Mx+N : M\in GL{\K^n}, N\in \K^n\}
 \]
 e l'operazione di composizione $\circ$: la coppia $(\Aff{\K^n},\circ))$ si dice gruppo delle affinità di $\K^n$.
 \end{definition}
 
 \begin{remark}
 la mappa $\phi:\K^n\rightarrow \K^{n+1}$, $x\mapsto
 \left(\begin{smallmatrix}
	x \\
	1
	\end{smallmatrix}\right)$
è un isomorfismo affine tra $\K^n$ e
$H=\{x\in \K^{n+1} : x_{n+1}=0\}$ il quale è sottospazio affine di $\K^{n+1}$.
 \end{remark}
 
 Vogliamo ora far vedere che questo isomorfismo è più profondo di quanto non appaia a prima vista. In particolare cercheremo di studiare
 la relazione tra le trasformazioni affini in $\K^n$ e le trasformazioni lineari in $\K^{n+1}$ arrivando ad identificare
 $\Aff{\K^n}$ con un sottogruppo di $GL(n,\K)$.
 
 \begin{definition}
  $G(H)=\{g\in GL(n+1,\K):g(H)=H\}$.
 \end{definition}
 
 \begin{remark}
 \`E immediato verificare che  $G(H)$ è un sottogruppo di  $GL(n+1,\K)$,
 dato che composizione ed inverse di mappe che conservano $H$ a loro volta lo conservano.
 \end{remark}
 
 \begin{remark}
 Se scomponiamo  $g\in G(H)$ in blocchi in questo modo:
 \[
	g = \left(
		\begin{array}{cc}
		M & N \\ 
		^tp & q
		\end{array}
	\right)
 \]
  dove $M\in M(n,\K);\  N,p\in \K^n,\ q \in \K$, e se vogliamo che  $g$ fissi $H$
 allora si deve avere che la $n+1$ -esima componente di
 $g\left(\begin{smallmatrix}
	x \\
	1
	\end{smallmatrix}
	\right)$
 sia ancora $1$,
 cioè che  $\forall x\in \K^n\  ^tpx+q=1$, cioè che $p=0, q=1$.
 \end{remark}
 
 \begin{proposition}
 sia  $\Phi:\Aff{\K^n}\rightarrow G(H)$ definito da
 \[
	\Phi(x\mapsto Mx+N)=\left(
		\begin{array}{cc}
		M & N \\ 
		0 & 1
		\end{array}
	\right)
 \]
 è un isomorfismo di gruppi.
 \end{proposition}
 
 \begin{proof}
 Per prima cosa è immediato vedere che $\Phi$  è iniettivo dato che il nucleo contiene solo l'identità.
 La surgettività ci è fornita dalla precedente osservazione. Rimane da far vedere che questo operatore commuta con la composizione.
 Date due affinità $a:x\mapsto Mx+N$ e $a':x\mapsto M'x+N',$ si ha che
 \[
	a\circ a':x\mapsto MM'x+ MN'+N
 \]
 quindi
 \[
	\Phi(a)\circ\Phi(a')=\left(
		\begin{array}{cc}
		MM' & MN'+N \\ 
		0 & 1
		\end{array}
	\right)=\Phi(a\circ a').
 \]
 \end{proof}
 