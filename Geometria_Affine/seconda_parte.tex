\section{Equivalenza affine}

\begin{definition}[Equivalenza rispetto a $G$]
	Se $G$ è un gruppo di trasformazioni di $\K^n$ e $F_1$ e $F_2$ sono sottoinsiemi di $\K^n$ allora $F_1$ e $F_2$ si dicono equivalenti per 
	$G$ se $\exists g\in G$ tale che $g(F_1)=F_2$.
\end{definition}

In particolare:

\begin{definition}
	$F_1,F_2\in\K^n$ si dicono affinemente equivalenti se $\exists g \in \Aff{\K^n}$ tale che $g(F_1) = F_2 .$
	Analogamente $F_1 , F_2 \in \K^n$ si dicono metricamente equivalenti se $\exists g \in \Isom{\K^n}$ tale che $g(F_1) = F_2 .$
\end{definition}
\begin{remark}
	In generale, data una classe di oggetti $X$ e un gruppo $G$ che agisce su $X$ (per esempio mediante applicazione a sinistra: $gx=g(x)$),
	otteniamo naturalmente una relazione di equivalenza dentro $X$ data da $x_1\sim x_2$ quando esiste $g\in G$ tale che $ x_1 = gx_2$.
	Le classi di equivalenza sono dette orbite dell'azione di $G$ in $X$. Studiamo in questi appunti le classi così ottenute dall'azione
	di particolari applicazioni di uno spazio vettoriale sull'insieme di ipersuperfici affini.
\end{remark}




\begin{definition}[Supporto di un polinomio]
	Dato $g\in \K[x_1,\dots,x_n]$ chiamiamo supporto di $g$ in $\K^n$ l'insieme $V(g)=\{x\in \K^n : g(x)=0\}$.
\end{definition}

\begin{definition}[Polinomi proporzionali]
	Dati $g_1,g_2\in \K[x_1,\dots,x_n]$ diciamo che sono proporzionali $(g_1\sim g_2)$ se $\exists \alpha \in \K^{*}$ tale che $g_1=\alpha g_2$.
\end{definition}
\begin{remark}
	\'E facile verificare che $\sim$ è una relazione di equivalenza
	(discende banalmente dal fatto che $\K^{*}$ è un gruppo rispetto alla moltiplicazione).
\end{remark}

\begin{definition}[Ipersuperficie affine]
	Ogni classe di equivalenza di polinomi $[g]$ in $\K[x_1,\dots,x_n]$ rispetto a $\sim$ è detta ipersuperficie.
	Inoltre $g(x)=0$ è detta equazione dell'ipersuperficie
\end{definition}

\begin{definition}[Ipersuperfici affinemente equivalenti]
	Date due ipersuperfici affini $[g],[h]$, si dicono affinemente equivalenti se esiste $\Psi\in \Aff{\K^n}$ tale che $[g]=[f\circ\Psi]$.
	In tal caso indicheremo $g\affeq h$
\end{definition}
\begin{remark}
	Useremo anche la notazione $\Psi^{-1}[g]$ per una ipersuperficie equivalente a $[g]$ mediante l'affinità $\Psi \in \Aff{\K^n}$.
\end{remark}
\begin{remark}
	Anche in questo caso è facile verificare che la relazione $\affeq$ è di equivalenza, dato che $\Aff{\K^n}$ è un gruppo rispetto alla
	composizione.
\end{remark}
\begin{remark}
	Notiamo che $[g]\affeq [h]\Rightarrow \text{deg}(g)=\text{deg}(h)$,
	poiché il grado di $g(\Psi(x))$ non può essere più grande di quello di $g(x)$, e il viceversa vale per l'invertibilità di $\Psi$.
\end{remark}

A questo punto passiamo allo studio di alcune ipersuperfici affini tra le più importanti. In particolare tratteremo il caso in cui le 
ipersuperfici sono di primo grado e quello in cui sono di grado due (dette quadriche) nei casi particolari in cui $\K=\R$ oppure $\K=\C$.

\begin{proposition}[Grado $1$]
	Tutti i polinomi di primo grado sono affinemente equivalenti;
	ossia esiste un'unica classe di equivalenza per le ipersuperfici di primo grado.
\end{proposition}

\begin{proof}
	Notiamo preliminarmente che possiamo denotare un polinomio di primo grado con $g(x)=\tra{A}x+b$ dove $A\in \K^n$, $A\neq 0$ e $b\in \K$.
	Consideriamo ora il polinomio $h(x)=\tra{A'}x+b'$ e cerchiamo $\Psi\in\Aff{\K^n}$ tale che $g\circ \Psi = h$, ovvero che
	\[
		\tra{A}(Mx+N)+b=\tra{A'}x+b'
	\]
	dove $M\in \GL{\K^n}$ e $N\in \K^n$. In particolare vogliamo che esistano tali $M$ e $N$.
	Ma è ovvio che esiste $M$ tale che $\tra{A}M=\tra{A'}$ e $N$ tale che $\tra{A}N+b=b'$ dato che $A$ è invertibile.
\end{proof}


\section{Quadriche reali e complesse}

Consideriamo ora il caso delle quadriche, ossia le ipersuperfici $[g]$ dove $g\in \K^n[x_1,\dots x_n]$
(qui $\K$ è il campo dei reali o dei complessi) e deg$(g)=2$. Cominciamo con
\begin{remark}
	Se $g$ è una quadrica, allora esistono $A\in M(n,\K)$ $A$ simmetrica e non nulla, $B\in \K^n$, $c\in \K$ tali che
	\[
		g(x) = \tra{x}Ax+2\tra{B}x+c.
	\]
\end{remark}

\begin{remark}\label{oss:isomf}
	Con un'altra osservazione possiamo ridurre ancora la notazione. Consideriamo $Q\in M(n+1,\K)$ associata al polinomio di secondo grado $g$:
	\[
		Q=\left(
			\begin{array}{cc}
			A & B\\
			\tra{B} & c
			\end{array}
		\right)
	\]
	e consideriamo anche $\tilde{x} =\left(\begin{smallmatrix}
x \\
1
\end{smallmatrix}
\right)$. Allora $g(x)=\tra{\tilde{x}}Q\tilde{x}$.
\end{remark}

\begin{definition}[Cono]
	Sia $X\subseteq V$ con $V$ uno spazio vettoriale. Diciamo che $X$ è un cono se $x\in X\Rightarrow \lambda x\in X$ per ogni $\lambda \in \K$.
\end{definition}
\begin{lemma}
	Se $g(x)=\tra{x}Qx$ allora $V(g)$ è un cono.
\end{lemma}
\begin{proof}
	Ciò è ovvio per l'omogeneità di $g(x)$.
\end{proof}

Data la \cref{oss:isomf}, possiamo equivalentemente parlare dello studio delle quadriche di $\K^n$ come delle quadriche omogenee in $\K^{n+1}$.
Consideriamo finalmente il cambiamento di forma delle quadriche via affinità.

\begin{lemma}\label{lem:cambioaffine}
	Data una quadrica $g$ a cui è associata la matrice $Q$ e una affinità $\Psi(x)=Mx+N$ a cui è possibile  associare la matrice
	$\tilde M_N = \left(\begin{smallmatrix}
M & N \\ 0 & 1
\end{smallmatrix}
\right)$ allora 
	$g\circ \Psi = \tra{\tilde M_N}Q \tilde M_N$.
\end{lemma}
\begin{proof}
	Basta fare un semplice conto: $g(\Psi(x))=\tra{\Psi(x)}Q \Psi(x) = \tra{x}\tra{\tilde M_N}Q \tilde M_N x$ da cui la tesi.
	Tuttavia se svolgiamo il conto tenendo presente che  $\tilde M_N$ e $Q$ sono matrici $n+1\times n+1$ otteniamo che
	\[
		\tra{\tilde M_N}Q \tilde M_N = 
		\left(
			\begin{array}{cc}
			\tra{M} & 0\\
			\tra{N} & 1
			\end{array}
		\right)
		\left(
			\begin{array}{cc}
			A & B\\
			\tra{B} & c
			\end{array}
		\right)
		\left(
			\begin{array}{cc}
			M & N\\
			0 & 1
			\end{array}
		\right) =
		\left(
			\begin{array}{cc}
			\tra{M}AM & \tra{M}AN+\tra{M}B\\
			\tra{(\tra{M}AN+\tra{M}B)}& \tra{N}AN+2\tra{B}N+c
			\end{array}
		\right)
	\]

\end{proof}

\begin{definition}[Conica a centro]
	Una conica $\mathfrak{C}=[g]$ è detta a centro se $\exists N\in \K^n$ tale che $g(x)=g(2N-x)$, ossia $\mathfrak{C}$ è invariante per la
	simmetria centrale rispetto a $N$. Una definizione analoga vale per le quadriche e per le ipersuperfici in generale.
\end{definition}

\begin{lemma}\label{lem:centroconica}
	Se $0$ è centro della conica $\mathfrak{C}=[g]$ su un campo a caratteristica diversa da due e $g=\tra{x}Ax+2Bx+C$, allora $B=0$.
\end{lemma}

\begin{proof}
	Per definizione di centro della conica si deve avere  $g(x)=g(-x)$ cioè $\tra{x}Ax+2Bx+C=\tra{-x}A(-x)+2B(-x)+C$, $4Bx=0$ che deve essere vero
	per ogni scelta di $x$, quindi si ha la tesi.
\end{proof}

\begin{remark}\label{lem:formulacentro}
 $N$ è centro di una quadrica se e solo se l'affinità $X\mapsto -X+2N$ lascia invariata $Q$, ossia, per la formula delle trasformazioni affini
 di una quadrica, se e solo se $-\tra IA(2N)-\tra IB=B$, $\tra (2N)A(2N)+2\tra B(2N)+c=c$ e$\tra(-I)A(-I)=A$. La prima condizione equivale a
 dire che $AN=-B$, e questo implica banalmente che anche la seconda condizione è vera, mentre la terza è banalmente sempre verificata.
 Dunque $N$ è centro della conica $Q$ se e solo se $AN=-B$.
\end{remark}
	
 \begin{lemma}\label{lem:centroaff}
  Se $g$ è una quadrica di centro $R$ e $\psi$ è un'affinità, allora $\psi^{-1}(R)$ è centro di $\psi^{-1}(g)$.
 \end{lemma}
 
 \begin{proof}
	Rappresentando il tutto in coordinate in $\K^{n+1}$, si ha che $g$ è rappresentata da $\left(\begin{smallmatrix}A & B \\ \tra B & C
	\end{smallmatrix} \right)$ mentre $\psi$ è rappresentata da $W=\left(\begin{smallmatrix} M & N \\ 0 & 1 \end{smallmatrix} \right)$.
	Allora per la formula della trasformazione per affinità di una quadrica la parte quadratica di $\psi^{-1}(g)$ è $\tra MAM$, mentre quella
	lineare è $\tra M(AN+B)$.
	Per la \cref{lem:formulacentro}, la tesi equivale a dire che $\tra MAM(M^{-1}(R-N))=-\tra M(AN+B)$, ossia, essendo $M$ invertibile,
	che $A(R-N)=-AN-B$, ossia che $AR=-B$, il che è vero perché $R$ è un centro di $g$.
 \end{proof}

\subsection{Classificazione affine delle coniche reali e complesse}

\begin{theorem}[Teorema di classificazione affine per le coniche reali e complesse]
	 Per le coniche complesse il rango di $A$ e quello di $Q$ insieme formano un sistema completo di invarianti per equivalenza affine.
	 
	 Invece per le coniche reali il rango e l'indice di Witt di $A$ e quelli di $Q$ insieme formano un sistema completo di invarianti
	 per equivalenza affine.
\end{theorem}
\begin{proof}
 Il fatto che quelli descritti siano effettivamente degli invarianti è vero in quanto $A$ e $Q$ con le affinità si trasformano per congruenza,
 e quelli descritti sono invarianti di congruenza, ed inoltre sono invarianti anche per il prodotto di costanti non nulle.
 Dunque rimane da dimostrare che questi sistemi di invarianti sono totali; per fare ciò verrà mostrato che ogni conica a seconda dei suoi
 invarianti è affinemente equivalente ad una forma canonica.
 \begin{description}
  \item[Coniche non a centro] Se la conica $g$ non è a centro, ciò significa che il sistema $AY=-B$ non ha soluzioni. Pertanto $A$ ha
  rango minore di due, e non potendo essere nulla (perché $g$ è una conica) ha rango uno.
  Dunque per il teorema di Sylvester esiste $M$ invertibile tale che $\tra MAM=\left(\begin{smallmatrix}\pm1 & 0 \\ 0 & 0 \end{smallmatrix}
  \right)$. Dunque l'affinità rappresentata da $\left(\begin{smallmatrix}M & 0 \\ \tra 0 & 1 \end{smallmatrix} \right)$ (più un eventuale cambio
  di segno consentito dalla definizione di conica) porta $Q$ in $$Q_1=\left(
  \begin{array}{ccc}
   1 & 0 & b_1\\
   0 & 0 & b_2\\
   b_1 & b_2 & d
  \end{array}
  \right)$$
  Se si cerca $N=\left(\begin{smallmatrix}\alpha\\ \beta \end{smallmatrix} \right)$ tale che la traslazione $x\mapsto x+N$ trasformi $Q_1$ in
  una matrice del tipo $$Q_2=\left(
  \begin{array}{ccc}
   1 & 0 & 0\\
   0 & 0 & c_2\\
   0 & c_2 & 0
  \end{array}
  \right)$$
  e si impone il sistema, si trova che esso ha come soluzione $\alpha=-b_1$ e $\beta=\frac{b_1^2-d}{2b_2}$ (che esiste perché $b_2$ è diverso
  da $0$, in quanto se non lo fosse il sistema $\left(\begin{smallmatrix} 1 & 0 \\ 0 & 0 \end{smallmatrix}\right)N=
  -\left(\begin{smallmatrix}b_1\\ b_2 \end{smallmatrix} \right) =
  -\left(\begin{smallmatrix} b_1\\ 0 \end{smallmatrix} \right)$ avrebbe soluzione, e dunque $Q_1$ avrebbe centro, e per la
  \cref{lem:formulacentro} $Q$ avrebbe centro, contro l'ipotesi).
  Con un'ulteriore affinità lineare ci si può ricondurre alla forma $$Q_3=\left(
  \begin{array}{ccc}
   1 & 0 & 0\\
   0 & 0 & -\frac{1}{2}\\
   0 & -\frac{1}{2} & 0
  \end{array}
  \right)$$
  Dunque ogni conica non a centro, reale o complessa, è affinemente equivalente alla quadrica rappresentata da $Q_3$, ossia alla
  quadrica $x^2-y=0$. Le coniche appartenenti a tale classe di equivalenza affine sono dette parabole.
  \item[Coniche non a centro] Se la conica $g$ è a centro, con un'opportuna traslazione la si può trasformare il una conica con centro
  nell'origine, la quale per il \cref{lem:centroconica} è della forma
  $$Q_1=\left(\begin{array}{cc} A & 0 \\ 0 & d \end{array} \right) .$$
  Dato che le coniche sono definite a meno del prodotto per costanti non nulle, se $d\ne 0$ si può moltiplicare la conica per $d^{-1}$, e dunque
  ci si può ricondurre ai due casi in cui $d=0$ ed in cui $d=1$.
  Ma a questo punto dobbiamo dividere il nostro studio a seconda che ci si trovi nel campo reale od in quello complesso.
 \item[Coniche non a centro complesse] Nel campo complesso per il teorema di Sylvester $A$, non essendo nulla e dunque non potento avere rango
 zero, è congruente a $\left(\begin{smallmatrix} 1 & 0 \\ 0 & 0 \end{smallmatrix}\right)$ se ha rango uno e a
 $\left(\begin{smallmatrix} 1 & 0 \\ 0 & 1 \end{smallmatrix}\right)$ se ha rango due. Pertanto con un'affinità lineare si può trasformare $Q_1$
 in una delle seguenti matrici:
 $$\left( \begin{array}{ccc}
   1 & 0 & 0\\
   0 & 1 & 0\\
   0 & 0 & 1
  \end{array} \right) ,
  \left(\begin{array}{ccc}
   1 & 0 & 0\\
   0 & 1 & 0\\
   0 & 0 & 0\end{array}\right),
   \left(\begin{array}{ccc}
   1 & 0 & 0\\
   0 & 0 & 0\\
   0 & 0 & 1\end{array}\right)
   \text{ o }\left(\begin{array}{ccc}
   1 & 0 & 0\\
   0 & 0 & 0\\
   0 & 0 & 0\end{array}\right) $$
   Dunque abbiamo trovato cinque forme canoniche per le coniche complesse: queste quattro e la parabola.
 Dato che nessuna di queste matrici ha la stessa coppia $(rg(A), rg(Q))$, si è dimostrato che ogni conica complessa è affinemente
 equivalente ad una tra le quattro forme canoniche appena trovate od alla forma canonica delle parabole, e che le loro classi di equivalenza
 affine sono disgiunte.
 \item[Coniche reali a centro] In questo caso per il teorema di Sylvester $A$ è congruente ad una tra
 $\left(\begin{smallmatrix} 1 & 0 \\ 0 & 1 \end{smallmatrix}\right)$, $\left(\begin{smallmatrix} 1 & 0 \\ 0 & -1 \end{smallmatrix}\right)$,
 $\left(\begin{smallmatrix} -1 & 0 \\ 0 & -1 \end{smallmatrix}\right)$, $\left(\begin{smallmatrix} 1 & 0 \\ 0 & 0 \end{smallmatrix}\right)$ e
 $\left(\begin{smallmatrix} -1 & 0 \\ 0 & 0 \end{smallmatrix}\right)$. Dunque, notando che se $d=0$ si può cambiare segno alla matrice, con
 un'affinità lineare ed un eventuale cambio di segno si può trasformare $Q_1$ in una delle seguenti coniche:
 $$\left( \begin{array}{ccc}
   1 & 0 & 0\\
   0 & 1 & 0\\
   0 & 0 & 1
  \end{array} \right) ,
  \left( \begin{array}{ccc}
   1 & 0 & 0\\
   0 & 1 & 0\\
   0 & 0 & 0
  \end{array} \right) ,
  \left( \begin{array}{ccc}
   1 & 0 & 0\\
   0 & -1 & 0\\
   0 & 0 & 1
  \end{array} \right) ,
  \left( \begin{array}{ccc}
   1 & 0 & 0\\
   0 & -1 & 0\\
   0 & 0 & 0
  \end{array} \right) ,
  \left( \begin{array}{ccc}
   1 & 0 & 0\\
   0 & 1 & 0\\
   0 & 0 & -1
  \end{array} \right) ,$$
  $$\left( \begin{array}{ccc}
   1 & 0 & 0\\
   0 & 0 & 0\\
   0 & 0 & 1
  \end{array} \right) ,
  \left( \begin{array}{ccc}
   1 & 0 & 0\\
   0 & 1 & 0\\
   0 & 0 & 0
  \end{array} \right) \text{ o }
  \left( \begin{array}{ccc}
   1 & 0 & 0\\
   0 & 0 & 0\\
   0 & 0 & -1
  \end{array} \right)$$
 \end{description}
 Dunque abbiamo trovato nove forme canoniche per le coniche reali (queste otto e la parabola) e dato che nessuna di esse ha la stessa quarterna
 $(rg(A),rg(Q),w(A),w(Q))$ esse sono rappresentanti di nove classi di equivalenza affine distinte di coniche reali.
\end{proof}
 Le cinque classi di equivalenza affine delle coniche complesse sono riassunte dalla seguente tabella:\\
 \vspace{5mm}\\
 \begin{tabular}{|c|c|c|c|c|}
 \hline
  Nome & Equazione della & Matrice della & rg(A) & rg(Q)\\
   &  forma canonica & forma canonica & & \\
  \hline
  & & & & \\
  Parabola & $x^2-y=0$ & $\left(\begin{smallmatrix} 1 & 0 & 0 \\ 0 & 0 & -\frac{1}{2}\\ 0 & -\frac{1}{2} & 0 \end{smallmatrix}\right)$ & 1 & 3 \\
  (non pervenuto) & $x^2+y^2+1=0$ & $\left(\begin{smallmatrix} 1 & 0 & 0 \\ 0 & 1 & 0 \\ 0 & 0 & 1 \end{smallmatrix}\right)$ & 2 & 3 \\
  Coppia di rette incidenti & $x^2+y^2=0$ & $\left(\begin{smallmatrix} 1 & 0 & 0 \\ 0 & 1 & 0 \\ 0 & 0 & 0 \end{smallmatrix}\right)$ & 2 & 2 \\
  Coppia di rette parallele & $x^2+1=0$ & $\left(\begin{smallmatrix} 1 & 0 & 0 \\ 0 & 0 & 0 \\ 0 & 0 & 1 \end{smallmatrix}\right)$ & 1 & 2 \\
  Retta doppia & $x^2=0$ & $\left(\begin{smallmatrix} 1 & 0 & 0 \\ 0 & 0 & 0 \\ 0 & 0 & 0 \end{smallmatrix}\right)$ & 1 & 1 \\
  \hline
 \end{tabular}
 
 \vspace{5mm}
 Invece le nove classi di equivalenza affine delle coniche reali sono le seguenti:\\
  \vspace{0mm}\\
 \begin{tabular}{|c|c|c|c|c|c|c|}
  \hline
  Nome & Equazione della & Matrice della & rg(A) & rg(Q) & w(A) & w(Q)\\
   &  forma canonica & forma canonica & & & & \\
  \hline
  & & & & & & \\
  Parabola&$x^2-y=0$& $\left(\begin{smallmatrix}1 & 0 & 0 \\ 0 & 0 &-\frac{1}{2}\\ 0 &-\frac{1}{2} & 0\end{smallmatrix}\right)$& 1 & 3 & 1 &1\\
  Ellisse immaginaria&$x^2+y^2+1=0$&$\left(\begin{smallmatrix} 1 & 0 & 0 \\ 0 & 1 & 0 \\ 0 & 0 & 1 \end{smallmatrix}\right)$&2&3&0&0\\
  Ellisse reale&$x^2+y^2-1=0$&$\left(\begin{smallmatrix} 1 & 0 & 0 \\ 0 & 1 & 0 \\ 0 & 0 & -1 \end{smallmatrix}\right)$&2&3&0&1\\
  Iperbole&$x^2-y^2-1=0$&$\left(\begin{smallmatrix} 1 & 0 & 0 \\ 0 & -1 & 0 \\ 0 & 0 & -1 \end{smallmatrix}\right)$&2&3&1&1\\
  Rette complesse incidenti&$x^2+y^2=0$&$\left(\begin{smallmatrix} 1 & 0 & 0 \\ 0 & 1 & 0 \\ 0 & 0 & 0 \end{smallmatrix}\right)$&2&2&0&1\\
  Rette incidenti&$x^2-y^2=0$&$\left(\begin{smallmatrix} 1 & 0 & 0 \\ 0 & -1 & 0 \\ 0 & 0 & 0 \end{smallmatrix}\right)$&2&2&1&2\\
  Rette complesse parallele&$x^2+1=0$&$\left(\begin{smallmatrix} 1 & 0 & 0 \\ 0 & 0 & 0 \\ 0 & 0 & 1 \end{smallmatrix}\right)$&1&2&1&1\\
  Rette parallele&$x^2-1=0$&$\left(\begin{smallmatrix} 1 & 0 & 0 \\ 0 & 0 & 0 \\ 0 & 0 & -1 \end{smallmatrix}\right)$&1&2&1&2\\
  Retta doppia&$x^2=0$&$\left(\begin{smallmatrix} 1 & 0 & 0 \\ 0 & 0 & 0 \\ 0 & 0 & 0 \end{smallmatrix}\right)$&1&1&1&2\\
  \hline
 \end{tabular}
 
 \subsection{Classificazione affine delle quadriche reali e complesse}
 \begin{theorem}[Teorema di classificazione affine per le quadriche reali e complesse]
 Ogni quadrica complessa è affine ad una delle seguenti (dove $r=rg(A)$):
 \begin{itemize}
  \item Se è a centro a $x_1^2+\ldots+x_r^2+d=0$, dove $d$ è uguale a zero o ad uno, ossia alla quadrica rappresentata dalla matrice
  $$\left( \begin{array}{cc|c}
   I_r & 0 & 0\\
   0 & 0_{n-r} & 0\\
   \hline
   0 & 0 & d
  \end{array} \right)$$
  \item Se non è a centro a $x_1^2+\ldots+x_r^2-x_n=0$ con $r<n$, ossia alla quadrica rappresentata dalla matrice
  $$\left( \begin{array}{cccc|c}
   I_r & & 0 & & 0\\
   & & & & 0\\
   0 & & 0_{n-r} & & \vdots\\
   & & & & -\frac{1}{2}\\
   \hline
   0 & 0 &\dots & -\frac{1}{2} & 0
  \end{array} \right)$$
 \end{itemize}
 Inoltre la coppia $rg(A), rg(Q))$ è un invariante totale di coniugio.
 
 Invece ogni quadrica reale è affine ad una delle seguenti:
 \begin{itemize}
  \item Se è a centro a $x_1^2+\ldots+x_p^2-x_{p+1}^2-\ldots-x_r+d=0$, dove $d$ è uguale a zero o ad uno, ossia alla quadrica
  rappresentata dalla matrice $$\left( \begin{array}{ccc|c}
   I_p & 0 & 0 & 0 \\
   0 & -I_{r-p} & 0 & 0\\
   0 & 0 & 0_{n-r} & 0 \\
   \hline
   0 & 0 & 0 & d
  \end{array} \right)$$
  \item Se non è a centro a $x_1^2+\ldots+x_p^2-x_{p+1}^2-\ldots-x_r^2-x_n=0$ con $r<n$ e $p\ge r-2p$, ossia alla quadrica rappresentata dalla
  matrice
  $$\left( \begin{array}{ccccc|c}
   I_p & 0 & & 0 & & 0\\
   0 & -I_{r-p} & & 0 & & 0 \\
   & & & & & 0\\
   0 & 0 & & 0_{n-r} & & \vdots\\
   & & & & & -\frac{1}{2}\\
   \hline
   0 & 0 & 0 &\dots & -\frac{1}{2} & 0
  \end{array} \right)$$
 \end{itemize}
 \end{theorem}
 \begin{proof}
  Il caso delle quadriche a centro è del tutto simile a quello delle analoghe coniche.
  Invece per le quadriche non a centro per il teorema di Sylvester esiste una trasformazione lineare che le trasforma quelle complesse in
  coniche del tipo
  $$\left( \begin{array}{ccc|c}
   I_r & 0 & & \\
   & & & Z\\
   0 & 0_{n-r} & & \\
   \hline
    & \tra Z & & c
  \end{array} \right),
  \text{ mentre trasforma quelle reali in }
  \left( \begin{array}{ccc|c}
   I_p & 0 & 0 & \\
   0 & -I_{r-p} & 0 & Z\\
   0 & 0 & 0_{n-r} & \\
   \hline
   & \tra Z & & c
  \end{array} \right),$$
  e con un cambio di segno ci si può ricondurre al caso $r-2p\ge p$.
  Qui considereremo solo le quadriche reali, in quanto la dimostrazione nel caso complesso è analoga a quello reale con $r=p$.
  $$ \text{Se } Z=\left( \begin{array}{c}
                         z_1\\
                         \vdots\\
                         z_n
                        \end{array}\right) \text{allora la traslazione } x\mapsto x-
                        \left( \begin{array}{c}
                         z_1\\
                         \vdots\\
                         z_p\\
                         -z_{p+1}\\
                         \vdots\\
                         -z_r\\
                         0\\
                         \vdots\\
                         0
                        \end{array}\right) \text{ porta la quadrica in una del tipo}$$
$$\left( \begin{array}{ccc|c}
   I_p & 0 & 0 & 0\\
   0 & -I_{r-p} & 0 & 0 \\
   0 & 0 & 0_{n-r} & W\\
   \hline
   0 & 0 & \tra W & c
  \end{array} \right) \text{ con } W=
  \left( \begin{array}{c}
  a_{r+1}\\
  \vdots\\
  a_n
  \end{array}\right)\ne 0 \text{ (perché altrimenti la quadrica avrebbe centro).}$$
  Ora cerchiamo un'affinità del tipo $$\left( \begin{array}{ccc|c}
   0 & 0 & 0 & 0\\
   0 & 0 & 0 & 0 \\
   0 & 0 & M & N\\
   \hline
   0 & 0 & 0 & 1
  \end{array} \right)$$
  che porti la conica alla forma canonica. Affinché ciò accada dev'essere che
  $$\left( \begin{array}{ccc|c}
   0 & 0 & 0 & 0\\
   0 & 0 & 0 & 0 \\
   0 & 0 & \tra M & 0\\
   \hline
   0 & 0 & \tra N & 1
  \end{array} \right)
  \left( \begin{array}{ccc|c}
   I_p & 0 & 0 & 0\\
   0 & -I_{r-p} & 0 & 0 \\
   0 & 0 & 0_{n-r} & W\\
   \hline
   0 & 0 & \tra W & c
  \end{array} \right)
  \left( \begin{array}{ccc|c}
   0 & 0 & 0 & 0\\
   0 & 0 & 0 & 0 \\
   0 & 0 & M & N\\
   \hline
   0 & 0 & 0 & 1
  \end{array} \right)=$$
  $$=\left( \begin{array}{ccc|c}
   0 & 0 & 0 & 0\\
   0 & 0 & 0 & 0 \\
   0 & 0 & 0 & \tra MW\\
   \hline
   0 & 0 & \tra WM & 2\tra NW+c
  \end{array} \right)=
  \left( \begin{array}{ccccc|c}
   I_p & 0 & & 0 & & 0\\
   0 & -I_{r-p} & & 0 & & 0 \\
   & & & & & 0\\
   0 & 0 & & 0_{n-r} & & \vdots\\
   & & & & & -\frac{1}{2}\\
   \hline
   0 & 0 & 0 &\dots & -\frac{1}{2} & 0
  \end{array} \right)$$
  Poiché $W\ne 0$, è banale che esista $N$ tale che $\tra NW=-c$, mentre se $(v_1,\ldots, v_{n-r-1})$ è una base di $W^{\perp}$
  allora la matrice $$M=\left( \begin{array}{c|c|c|c|c}
						v_1 & v_2 & \dots & v_{n-r-1} & -\frac{W}{2\tra WW}
                      \end{array} \right)
  \text{ è tale che } \tra MW=
  \left( \begin{array}{c}
          0\\
          0\\
          \vdots\\
          -\frac{1}{2}
         \end{array} \right).$$
  Dunque è stato dimostrato che ogni quadrica, reale o complessa, è affinemente equivalente ad una delle forme canoniche indicate.
  Il fatto che gli invarianti di cui si parla nella tesi siano effettivamente degli invarianti è stato già visto, ed è facile mostrare che
  ciascuna delle forme canoniche trovate abbia una stringa di invarianti diversa.
 \end{proof}

\begin{remark}
	A titolo di curiosità, si usa questa nomenclatura per la classificazione delle quadriche:
	\begin{description}
	 \item[paraboloidi] sono quadriche non a centro;
	 \item[ellissoidi] sono quadriche reali con indice di Witt nullo;
	 \item[iperboloidi] sono le altre quadriche reali;
	 \item[???] e chi più ne ha, più ne metta!
	\end{description}

\end{remark}


