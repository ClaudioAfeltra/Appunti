\documentclass[a4paper,12pt]{article} 
\usepackage{stilebase}

\newcommand{\Got}[1]{#1}
\newcommand{\got}[1]{{#1}}
\newcommand{\GOT}[1]{\Large\textbf{#1}\normalsize}
\newcommand{\vet}[1]{\overrightarrow{#1}}





\title{\bf Geometria affine}
\date{}
\author{}

\begin{document}

 \maketitle
 
 \section{Gruppi di trasformazioni}
 
 \Got{Sia} $\got{X}$ \Got{un  insieme non vuoto. Si definisce} $$\got{S(X)=\{f:X\to X\quad biunivoche\}}.$$ 
 \Got{Allora} $\got{(S(X),}\circ\got)$ \Got{è un gruppo.}
 
 \begin{definition}
 \Got{Si dice gruppo di trasformazioni di} $\got{X}$ \Got{ogni sottogruppo di} $\got{S(X).}$
 \end{definition}
 
 \begin{esempio}
 \Got{}$\got{GL(V)}$ \Got{(dove} $\got{V}$ \Got{è uno spazio vettoriale);} $\got{O(V,\phi)}$ \Got{con} $\got{\phi\in PS(V).}$
 \end{esempio}
 
 \subsection{Traslazioni}
 
 \Got{Esistono trasformazioni naturali e semplici, ma non lineari.}
 
 \begin{definition}
  \Got{Dato uno spazio vettoriale} $\got{V}$ \Got{ed un suo elemento} $\got{v}$ \Got{si dice traslazione di vettore} $\got{v}$ \Got{la funzione}
 \begin{eqnarray*}
	\got{\tau_v:V} & \to & \got{V} \\
	\got{w} & \mapsto & \got{w+v}
 \end{eqnarray*}
 \end{definition}
 
 \begin{remark}
  \Got{}$\got{\tau_v}$ \Got{è lineare se e solo se} $\got{v=0}$.
 \end{remark}
 
 \begin{proposition}
 \Got{Sia} $\got{T(V)=\{\tau_v\;|\;v\in V\}}$. \Got{Allora} $\got{(T(V),}\circ\got{)}$
 \Got{è un gruppo abeliano di trasformazioni di} $\got{V}$, \Got{ed è isomorfo a} $\got{(V,+)}$.
 \end{proposition}
 
 \begin{proof}
\Got{La proprietà associativa vale perché l'insieme è formato da funzioni.}
 $\got{\tau_0}$ \Got{è evidentemente l'elemento neutro, e} $\got{\forall v\in V\;\; \tau_v}\circ\got{\tau_{-v}=\tau_{-v}}\circ\got{\tau_v=id=\tau_0}$, 
 \Got{dunque} $\got{\tau_v^{-1}=\tau_{-v}}$.
 \Got{Inoltre} $\got{\forall u,v,w\in V\quad (\tau_v}\circ\got{\tau_w)(u)=\tau_v(\tau_w(u))=\tau_v(u+w)=u+w+v=\tau_w(\tau_v(u))=}$\\
 $\got{=(\tau_w}\circ\got{\tau_w)(v)=(\tau_w}\circ\got{\tau_v)(u)}$, \Got{dunque } $\got{\tau_v}\circ\got{\tau_w(u)=\tau_w}\circ\got{\tau_v}$,
 \Got{ pertanto il gruppo è abeliano.
 Per la seconda parte della tesi, la funzione}
 \begin{eqnarray*}
	\got{(V,+)} & \to & \got{(T(V))} \\
	\got{v} & \mapsto & \got{\tau_v}
 \end{eqnarray*}
 \Got{è un isomorfismo di gruppi, perché è ovviamente biunivoca, e } $\got{\tau_{v+w}=\tau_v}\circ\got{\tau_w}$.
  \end{proof}
  
	\subsection{Azione di un gruppo}
 
 \Got{Dato un gruppo} $\got{G}$ \Got{ed un insieme} $\got{X}$ \Got{si dice azione di} $\got{G}$ \Got{su} $\got{X}$ \Got{un omomorfismo}
 $\psi:\got{G\to S(x)}$.
 \Got{L'azione si dice transitiva se } $\got{\forall x,y\in X\;\;\exists g\in G}$ \Got{tale che } $\got{\psi(g)(x)=y}$.
 
 \begin{remark}
 \Got{}$\got{T(V)}$ \Got{agisce su} $\got{V}$ \Got{in modo transitivo.
 Infatti se} $\got{v,w\in V,\quad\tau_{w-v}(v)=w}$.
 \end{remark}
 
	\subsection{Gruppo delle isometrie}
 
 \Got{Sia} $\got{(V,\phi)}$ \Got{uno spazio euclideo.} $\got{\phi}$ \Got{induce una distanza} $\got{d}$ \Got{su} $\got{V}$. \Got{Allora si pone}
 $$\got{\Isom{V,d}=\{f:V\to V\;|\;\forall P,Q\in V\quad d(P,Q)=d(f(P),f(Q))\}}.$$
 
 \begin{remark} $\got{\Isom(V,d)}$ \Got{gode delle seguenti proprietà:}
 \begin{itemize}
	\item $\got{O(V,\phi)\subseteq \Isom(V,d)}$;
	\item $\got{T(V)\subseteq \Isom(V,d)}$;
	\item $\got{\forall v\in V,\forall f\in O(V,\phi)\quad\tau_v}\circ\got{ f\in \Isom(V,d)}$
 \end{itemize}
 \end{remark}
 
 \begin{lemma}
 \Got{}$\got{\forall v\in V, \forall f\in GL(V)}$ \Got{(e dunque in particolare per} 
 $\got{f\in O(V,\phi))}$\\
 $\got{f}\circ\got{\tau_v=\tau_{f(v)}}\circ\got{ f}$ \Got{(e pertanto in generale} $\got{f}$ \Got{e} $\got{\tau_v}$ \Got{non commutano)}.
 \end{lemma}
 
 \begin{proof}
 \Got{}$\got{(f}\circ\got{\tau_v)(x)=f(x+v)=f(x)+f(v)=\tau_{f(v)}(f(x))=(\tau_{f(v)}}\circ\got{ f)(x)}$.
  \end{proof}
 
 \begin{proposition}
 \Got{}$\got{\{\tau_v}\circ\got{ f\;|\;v\in V,f\in O(V,\phi)\}}$ \Got{è un gruppo rispetto alla composizione.}
 \end{proposition}
 
 \begin{proof}
 \Got{Poiché}  $\got{(\tau_v}\circ\got{ f)}\circ\got{(\tau_w}\circ\got{ g)=\tau_v}\circ\got{(f}\circ\got{\tau_w)}\circ\got{ g=
 \tau_v}\circ\got{\tau_{f(w)}}\circ\got{ f}\circ\got{ g}$, \Got{tale insieme è chiuso rispetto alla composizione.
 Inoltre per questo è vero che} $\got{(\tau_v}\circ\got{ f)}\circ\got{(\tau_w}\circ\got{ g)=id}$ \Got{se e solo se} $\got{\tau_v}\circ\got{\tau_{f(w)}}\circ\got{ f}\circ\got{ g}$,
 \Got{la qual cosa è senza dubbio vera se} $\got{f(w)=-v}$ \Got{ e}  $\got{f}\circ\got{ g=id}$, \Got{ossia se} $\got{w=-f^{-1}(v)}$ \Got{ e }
 $\got{g=f^{-1}}$.
  \end{proof}
 
 \begin{remark}
 \Got{Analogamente si dimostra che anche} $\got{\{\tau_v}\circ\got{ f\;|\;v\in V,f\in GL(V)\}}$ 
 \Got{è un gruppo di trasformazioni.}
 \end{remark}
 
 \begin{theorem}
 \label{thm:isometrie}
 \Got{}$\got{\Isom(V,\phi)=\{\tau_v}\circ\got{ f\;|\;v\in V,f\in O(V,\phi)\}}$.
 \end{theorem}
 
 \begin{proof}
 \Got{Il fatto che il membro destro dell'uguaglianza sia incluso in quello sinistro è ovvio.\\
 Viceversa, se} $\got{f\in \Isom(V,\phi)}$ \Got{ e } $\got{f(0)=v}$, \Got{allora se } $\got{\tau_{-v}}\circ\got{ f,\; g\in \Isom(V,\phi)}$ \Got{e}
 $\got{g(0)=0}$. \Got{Dunque, per un teorema visto,} $\got{g\in O(V,\phi)}$, \Got{e pertanto} $\got{f=\tau_v}\circ\got{ g}$.
  \end{proof}
 
 \begin{corollary}
\Got{}$\got{\Isom(V,\phi)}$ \Got{è un gruppo di trasformazioni.}
 \end{corollary}
 
	\section{Isometrie di $\R^n$}
 
 \Got{Si consideri adesso come caso particolare} $\got{\R^n}$ \Got{dotato del prodotto scalare ordinario.
 Il \cref{thm:isometrie} allora implica che} $$\got{\Isom(\R^n)=\{X\mapsto AX+B|A\in O(n),B\in \R^n\}}.$$
 \Got{Pertanto se} $\got{f\in \Isom(\R^n)}$ \Got{allora} $$\got{Fix(f)=\{X\in \R^n|AX+B=X\}=\{X\in \R^n|(A-I)X=-B\}},$$
 \Got{quindi o} $\got{Fix(f)}$ \Got{è vuoto, o è un sottospazio affine di} $\got{\R^n}$ \Got{di giacitura}
 $$\got{\{X\in \R^n|(A-I)X=0\}=Fix(A).}$$
 
	\subsection{Simmetrie}
 
 $\got{f\in \Isom(\R^n)}$ \Got{si dice simmetria se} $\got{f^2=id}$.
 
 \begin{proposition}
 \Got{Sia} $\got{f}$ \Got{una simmetria di} $\got{\R^n}$ \Got{tale che} $\got{f(X)=AX+B}$, \Got{con} $\got{A\in O(n}$).
 \Got{Allora:}
 \begin{enumerate}[label=\bf\Roman*)]
	\item $\got{A^2=I}$ \Got{(e dunque} $\got{A}$ \Got{è diagonalizzabile) e} $\got{f(B)=AB+B=0}$;
	\item $\got{\frac{B}{2}\in Fix(f)}$, \Got{e dunque} $\got{Fix(f)=Fix(A)+\frac{B}{2}}$;
	\item $\got{B}$ \Got{è ortogonale a} $\got{Fix(A)}$.
 \end{enumerate}
 \end{proposition}
 
 \begin{proof}
 \begin{enumerate}[label=\bf\Roman*)]
 \item $\got{\forall X\in \R^n,  X=f^2(X)=A(AX+B)+B=A^2X+AB+B}$, \Got{e dunque}\\
 $\got{A^2=I}$ \Got{e} $\got{f(B)=AB+B=0}$.
 \item $\got{f(\frac{B}{2})=\frac{AB}{2}+B=\frac{AB+B}{2}+\frac{B}{2}=\frac{B}{2}}$.
 \item Poiché $\got{A^2=I},\quad A$ \Got{è diagonalizzabile con autovalori} $\got{1}$ \Got{e} $\got{-1}$, \Got{pertanto}\\
 $\got{R^n=V(1,A)\oplus V(-1,A)}$. \Got{Per definizione} $\got{Fix(A)=V(1,A)}$. \Got{Vogliamo dimostrare che} $\got{Fix(A)^{\bot}=V(-1,A)}$.
 \Got{Se} $\got{x\in V(-1,A)}$ \Got{e} $\got{y\in V(1,A)}$, \Got{allora} $$\got{\langle x,y\rangle=^txy=^t(-Ax)Ay=-^tx^tAAy=-^txy=
 -\langle x,y\rangle},$$ \Got{pertanto} $\got{\langle x,y\rangle=0}.$
 \Got{Dunque} $\got{V(-1,A)\subseteq Fix(A)}$. \Got{Ma entrambi hanno dimensione}\\
 $\got{n-dim(V(1,A))}$, \Got{dunque coincidono.
 Ma per il punto \bf{I}} $\got{B\in V(-1,A)}$.
 \end{enumerate}
  \end{proof}
 
	\subsection{Riflessioni}
 
 $\got{f\in \Isom(\R^n)}$ \Got{viene detta riflessione se} $\got{f^2=id}$ \Got{e} $\got{Fix(f)}$ 
 \Got{è un iperpiano affine (ossia se ha dimensione} $\got{n-1}$).
 
 \begin{remark}
 \Got{Se} $\got{f(X)=AX+B}$ \Got{è una riflessione,} $\got{dim(Fix(f))=n-1}$, \Got{per cui}\\
 $\got{A\in O(n)}$ \Got{induce una riflessione lineare rispetto alla giacitura di} $\got{Fix(f)}$. \Got{In particolare} $\got{det\,A=-1}$.
 \end{remark}
 
 \begin{exercise}
\Got{In} $\got{\R^2}$ \Got{siano} $\got{r_1}$ \Got{e} $\got{r_2}$ \Got{due rette passanti per l'origine. Sia} $\got{\alpha}$
 \Got{l'angolo (considerato in senso antiorario) tra} $\got{r_1}$ \Got{e} $\got{r_2}$. \Got{Siano} $\got{\rho_1}$ \Got{e} $\got{\rho_2}$
 \Got{le riflessioni del piano di assi rispettivamente} $\got{r_1}$ \Got{e} $\got{r_2}$.
 \Got{Si dimostri che} $\got{\rho_2}\circ\got{\rho_1}$ \Got{è la rotazione antioraria del piano avente come centro l'origine e di angolo} 
 $\got{2\alpha}$.
 \end{exercise}
 
 \begin{exercise}
 \Got{Studiare la composizione di due riflessioni distinte di} $\got{\R^n}$.
 \end{exercise}
 
 \begin{soluzione}
\Got{Siano} $\got{\rho_1(X)=A_1X+B_1}$ \Got{e} $\got{\rho_2(X)=A_2X+B_2}$, $\got{Fix(\rho_1)=H_1}$ \Got{e}\\
 $\got{Fix(\rho_2)=H_2}$. 
 \Got{Sappiamo che} $\got{A_1^2=A_2^2=I}$ \Got{e} $\got{A_1B_1+B_1=A_2B_2+B_2=0}$.
 
 \begin{description}
  \item[I caso:] \Got{}$\got{H_1}$ \Got{e} $\got{H_2}$ \Got{sono paralleli (ossia hanno la stessa giacitura).
 Allora} $\got{Fix(A_1)=Fix(A_2)}$. \Got{Pertanto} $\got{A_1}$ \Got{e} $\got{A_2}$ \Got{inducono una riflessione rispetto allo stesso piano.
 Dunque} $\got{A_1=A_2=A}$.
 \Got{Ma allora} $\got{(\rho_2}\circ\got{\rho_1)(X)=A(AX+B_1)+B_2=A^2X+AB_1+B_2=X+(AB_1+B_2)=}$\\
 $\got{=X+(B_2-B_1)}$
 \Got{Di conseguenza} $\got{\rho_2}\circ\got{\rho_1}$ \Got{è una traslazione di} $\got{B_2-B_1}$, \Got{che è il doppio della distanza tra}
 $\got{H_1}$ \Got{e} $\got{H_2}$.
 \item[II caso: ] \Got{}$\got{H_1}$ \Got{e} $\got{H_2}$ \Got{non sono paralleli.}
 \Got{In tal caso} $\got{H_1\cap H_2}$ \Got{è un sottospazio affine di dimensione} $\got{n-2}$.
 \Got{Ma} $\got{H_1\cap H_2\subseteq Fix(\rho_1}\circ\got{\rho_2)}$.
 \Got{Inoltre la giacitura di} $\got{Fix(\rho_2}\circ\got{\rho_1)}$ \Got{è}\\
 $\got{Fix(A_2A_1)=V(1,A_2A_1)}$. \Got{Allora} $\got{L=V(1,A_2A_1)^{\bot}=V(-1,A_2A_1)}$ \Got{è invariante per} $\got{A_2A_1}$.
 \Got{Pertanto tutti i piani ortogonali a} $\got{H_1\cap H_2}$ \Got{(che hanno giacitura} $\got{L}$) \Got{sono invarianti per}
 $\got{\rho_2}\circ\got{\rho_1}$.
 \Got{Per l'esercizio precedente su tali piani} $\got{\rho_2}\circ\got{\rho_1}$ \Got{è composizione di riflessioni rispetto a rette incidenti,
 dunque è una rotazione.}
 \end{description}
 \end{soluzione}
 
 \begin{exercise}[Forma canonica per le simmetrie]
\Got{Dimostrare che se} $\got{f\in \Isom(\R^n)}$ \Got{è una simmetria e}
 $\got{k=dim(Fix(f)),}$ \Got{allora esiste una base ortonormale} $\got{B=\{v_1,\ldots,v_n\}}$ \Got{di} $\got{\R^n}$ \Got{rispetto alla quale}
 $\got{f}$ \Got{si scrive come}
 $$\got{f(X)}=\left(
 \begin{array}{cc}
  \got{I_k} & \got{0}\\
  \got{0} & \got{I_{n-k}}
 \end{array}
 \right)
 \got{X+\alpha v_n}.
 $$
 \end{exercise}
 
 \begin{theorem}
 \Got{Ogni} $\got{f\in \Isom(R^n)}$ \Got{è composizione di al più} $\got{n+1}$ \Got{riflessioni}
 \end{theorem}
 
 \begin{proof}
 \Got{Se} $\got{f(0)=0}$, \Got{allora} $\got{f\in O(n)}$, \Got{dunque è composizione di al più} $\got{n}$ 
 \Got{riflessioni.
 Se} $\got{f(0)=B\ne 0}$, \Got{sia} $\got{H=\{X\in R^n|\;d(X,0)=d(X,B)\}}$.
 $\got{H}$ \Got{è un iperpiano. Infatti} $$\quad\got{x\in H\Longleftrightarrow ||x||^2=||x-B||^2\Longleftrightarrow\langle x,x\rangle=
 \langle x-B,x-B\rangle\Longleftrightarrow\langle B,B\rangle-2\langle B,X\rangle=0},$$
 \Got{equazione che definisce un piano. Inoltre, poiché} $\got{\frac{B}{2}\in H}$, \Got{la giacitura di} $\got{H}$ \Got{è} 
 $\got{Span(B)^{\bot}}$.
 \Got{Sia} $\got{\rho_H}$ \Got{la riflessione rispetto all'iperpiano} $\got{H}$.
 \Got{Allora}
 
 \[ \left\{
 \begin{array}{l}
  \got{\rho_H}\circ\got{ f\in \Isom(R^n)} \\
  \got{(\rho_H}\circ\got{ f)(0)=\rho_H(B)=0}
 \end{array}
 \right.
 \Longrightarrow \got{\rho_H}\circ\got{ f\in O(n)\Longrightarrow \rho_H}\circ\got{ f}
 \]
 \Got{è composizione di al più} $\got{n}$ \Got{riflessioni, dunque} $\got{f=\rho_h}\circ\got{(\rho_H}\circ\got{ f)}$
 \Got{è composizione di al più} $\got{n+1}$ \Got{riflessioni}.
  \end{proof}
 
 \begin{definition}\Got{Un' isometria si dice diretta se è composizione di un numero pari di riflessioni, si dice inversa se
 è composizione di un numero dispari di riflessioni. Se} $\got{f(X)=AX+B,\quad f}$ \Got{è diretta se} $\got{det\; A=1}$,
 \Got{è inversa se} $\got{det\; A=-1}$.
 \end{definition}
 
 \begin{proposition}
 \Got{Se} $\got{f\in \Isom(R^n)}$ \Got{ha almeno un punto fisso, è composizione di al più} $\got{n}$ \Got{riflessioni}.
 \end{proposition}
 
 \begin{proof}
 \Got{Sia} $\got{f(0)=B\ne 0}$ \Got{(se} $\got{f(0)=0}$ \Got{il la tesi è manifesta) e sia} $\got{Q\in R^n}$
 \Got{tale che} $\got{f(Q)=Q}$. \Got{Sia inoltre} $\got{H}$ \Got{definito come nella dimostrazione precedente.
 Allora come prima} $\got{\rho_H}\circ\got{ f\in O(n)}$, \Got{e poiché} $\got{d(Q,0)=d(f(Q),f(0))=d(Q,B),\quad Q\in H}$, \Got{e pertanto}\\
 $\got{(\rho_h}\circ\got{ f)(Q)=Q}$.
 \Got{Perciò} $\got{dim(Fix(\rho_H}\circ\got{ f))\ge 1}$, \Got{e di conseguenza} $\got{\rho_h}\circ\got{ f}$ \Got{è composizione di al più}
 $\got{n-1}$
 \Got{riflessioni.
 Dunque} $\got{f}$ \Got{è composizione di al più} $\got{n}$ \Got{riflessioni.}
  \end{proof}

	\subsection{Classificazione delle isometrie di $\got{\R^2}$}
 
 \Got{Alcuni esempi di isometrie di} $\got{\R^2}$ \Got{sono:}
 \begin{itemize}
	\item \Got{traslazioni;}
	\item \Got{rotazioni;}
	\item \Got{riflessioni;}
	\item \Got{composizioni delle precedenti.}
 \end{itemize}
 
 \begin{definition}
 \Got{Si chiama glissoriflessione (o riflessione rotatoria) la composizione di una riflessione con una traslazione parallela
 all'asse della riflessione.}
 \end{definition}
 
 \begin{remark}
 \Got{La glissoriflessione è un'isometria inversa senza punti fissi.}
 \end{remark}
 
 \begin{theorem}[di classificazione delle isometrie piane]
 
\Got{Ogni isometria di} $\got{\R^2}$ \Got{è di uno dei seguenti tipi:}
 \begin{enumerate}[label=\bf\Roman*)]
  \item \Got{traslazione (isometria diretta senza punti fissi)};
  \item \Got{rotazione (isometria diretta con punti fissi)};
  \item \Got{riflessione (isometria inversa con punti fissi)};
  \item \Got{glissoriflessione (isometria inversa senza punti fissi).}
 \end{enumerate} 
 \end{theorem}
 \begin{proof}
 \Got{Sia} $\got{f\in \Isom(\R^2)}$. \Got{Allora} $\got{f}$ \Got{è composizione di} $\got{k}$ \Got{riflessioni, con}
 $\got{k\le 3}$.
 \Got{Se} $\got{k=0,\; f}$ \Got{è l'identità.
 Se} $\got{k=1,\; f}$ \Got{è una riflessione.
 Se} $\got{k=2}$, \Got{come abbiamo già visto} $\got{f}$ \Got{è una traslazione od una rotazione.
 Se} $\got{k=3,\; f=\rho_1}\circ\got{\rho_2}\circ\got{\rho_3}$. \Got{Sia} $\got{r_i=Fix(\rho_i)}$ \Got{(con} $\got{i=1,2,3}$).
 $\got{\rho_1}\circ\got{\rho_2}$ \Got{può essere una traslazione od una rotazione.}
 \begin{description}
 \item[I caso:] \Got{}$\got{\rho_1}\circ\got{\rho_2=\tau_v}$ \Got{è una traslazione.
 Allora sia} $\got{v=v_1+v_2}$, \Got{con} $\got{v_1//r_3}$ \Got{e} $\got{v_2\bot r_3}$.
 \Got{Quindi} $\got{f=\tau_v}\circ\got{\rho_3=\tau_{v_1}}\circ\got{(\tau_{v_2}}\circ\got{\rho_3)}$. \Got{Ma} $\got{\tau_{v_2}}\circ\got{\rho_3}$
 \Got{è una riflessione}
 $\got{\rho'_3}$ \Got{rispetto ad un asse} $\got{r'_3}$ \Got{parallelo a} $\got{r_3}$.
 \Got{Dunque} $\got{f=\tau_{v_1}}\circ\got{\rho'_3}$, \Got{che è una glissoriflessione se} $\got{v_1\ne 0}$, \Got{è una riflessione se} 
 $\got{v_1=0}$.
 \item[II caso:] \Got{}$\got{\rho_1}\circ\got{\rho_2}$ \Got{è una rotazione rispetto ad un punto} $\got{P}$ \Got{di angolo} $\got{\alpha}$.
 \Got{Vanno distinti due ulteriori sottocasi:}
 \begin{itemize}
  \item $\got{P\notin r_3}$. \Got{In tal caso la rotazione} $\got{R=\rho_1}\circ\got{\rho_2}$ \Got{è composizione di due riflessioni}
  $\got{\rho'_1}$
  \Got{e} $\got{\rho'_2}$, \Got{rispetto a rette} $\got{r'_1}$ \Got{e} $\got{r'_2}$ \Got{incidenti in} $\got{P}$,
  \Got{che formano un angolo di} $\got{\frac{\alpha}{2}}$ \Got{e con} $\got{r'_2}$ \Got{parallela a} $\got{r'_3}$.
  \Got{Allora} $\got{f=\rho_1}\circ\got{\rho_2}\circ\got{\rho_3=\rho'_1}\circ\got{\rho'_2}\circ\got{\rho_3}$. \Got{Ma}
  $\got{\rho'_2}\circ\got{\rho_3}$
  \Got{è una traslazione, dunque} $\got{f}$ \Got{è una glissoriflessione}.
  \item $\got{\got{P\in r_3}}$ \Got{(ossia} $\got{r_1,\; r_2}$ \Got{e} $\got{r_3}$ \Got{sono incidenti). Allora si può scrivere}
  $\got{\rho_1}\circ\got{\rho_2}$ \Got{come} $\got{\rho'_1}\circ\got{\rho_3}$, \Got{dove} $\got{\rho'_1}$ \Got{è la riflessione rispetto ad un
  opportuno asse passante per} $\got{P}$.
  \Got{Allora} $$\got{f=\rho_1}\circ\got{\rho_2}\circ\got{\rho_3=\rho'_1}\circ\got{\rho_3}\circ\got{\rho_3=\rho'_1,}$$ \Got{e dunque} $\got{f}$ \Got{è una rotazione.} 
 \end{itemize}
 \end{description}
  \end{proof}
 \begin{remark}
 \Got{}$\got{\rho_1}\circ\got{\rho_2}\circ\got{\rho_3}$ \Got{è una riflessione se e solo se i tre assi di riflessione sono paralleli
 o incidenti.}
 \end{remark}
 
  \subsection{Classificazione delle isometrie di $\got{\R^3}$}
  
 \Got{Alcuni esempi di isometrie di} $\got{\R^3}$ \Got{sono:}
 \begin{itemize}
  \item \Got{traslazioni (luogo dei punti fissi:} $\got{\emptyset}$);
  \item \Got{riflessioni (luogo dei punti fissi: un piano);}
  \item \Got{rotazioni (luogo dei punti fissi: una retta).}
 \end{itemize}
 \begin{remark}
 \Got{La composizione di tre riflessioni rispetto a tre piani distinti incidenti in un punto (ma non in una retta)
 è la simmetria rispetto a quel punto.
 In generale due isometrie elementari (traslazioni, riflessioni, rotazioni) non commutano, fuorché in alcuni casi particolari,
 cui si danno dei nomi:}
 \begin{enumerate}[label=\bf\Roman*)]
  \item  \Got{si chiama glissorotazione la composizione di una riflessione e di una traslazione parallela al piano di riflessione;}
  \item \Got{si chiama riflessione rotatoria la composizione di una riflessione e di una rotazione intorno ad una retta ortogonale
  al piano di riflessione (un caso particolare sono le simmetrie centrali);}
  \item \Got{si chiama avvitamento la composizione di una rotazione e di una traslazione parallela alla retta intorno a cui
  avviene la rotazione.}
 \end{enumerate}
 \end{remark}
 \begin{theorem}
 \Got{Tutte le isometrie di} $\got{\R^3}$ \Got{sono traslazioni, riflessioni, rotazioni, glissorotazioni, riflessioni rotatorie
 o avvitamenti.}
 \end{theorem}
 
  \section{Il gruppo di trasformazioni $\got{A(V)}$}
 \Got{Dato uno spazio vettoriale} $\got{V}$ \Got{si definisce} $\got{A(V)=\{\tau_v}\circ\got{ f\;|\; v\in V,\; f\in GL(V)\}}$.
 \Got{È stato già visto che} $\got{A(V)}$ \Got{è un gruppo di trasformazioni, e che coincide con il sottogruppo di} $\got{S(V)}$
 \Got{generato da} $\got{T(V)}$ \Got{e da} $\got{GL(V)}$.
 
 \vspace{0.5cm}
 
 \begin{definition}
 \Got{Dati un gruppo di trasformazioni} $\got{G}$ \Got{di un insieme} $\got{X}$ \Got{e un elemento} $\got{x}$ \Got{di}
 $\got{X}$ \Got{si chiama stabilizzatore di} $\got{x}$ \Got{l'insieme} $\got{St_x(G)=\{g\in G\;|\; g(x)=x\}}$.
 \end{definition}
 
 \begin{proposition}
 \begin{enumerate}[label=\Roman*)]
	 \item Se $\got{|V|>2}$ \Got{allora} $\got{St_v(GL(V))=GL(V)}$ \Got{se e solo se} $\got{v=0}$.
	 \item $\got{\forall v\in V\quad St_v(A(G))\cong GL(V)}$.
	\end{enumerate} 
 \end{proposition}
 
 \begin{proof}
 \Got{I) Ovviamente} $\got{St_0(GL(V))=GL(V)}.$
 \Got{D'altra parte se} $\got{v\ne 0}$ \Got{allora è banale trovare una funzione} $\got{f\in GL(V)}$ \Got{tale che} $\got{f(v)\ne v.}$
  
 
 \Got{II) Innanzitutto} $\got{\forall v,w\in V\;\; St_v(A(V))}$ \Got{e} $\got{St_w(A(V))}$ \Got{sono isomorfi.
 Infatti se}\\
 $\got{u=v-w}$ \Got{(sicché} $\got{\tau_u(w)=v)}$
 \begin{eqnarray*}
	\got{St_v(A(V))} &\to & \got{St_w(A(V))} \\
	\got{f} &\mapsto & \got{\tau_{-u}}\circ\got{ f}\circ\got{\tau_u}
 \end{eqnarray*}
 
 \Got{è un isomorfismo.
 Inoltre} $\got{\tau_v}\circ\got{ f\in St_0(A(V))}$ \Got{se e solo se} $\got{(\tau_v}\circ\got{ f)(0)=0,}$ \Got{cioè se e solo se} $\got{v=0}$,
 \Got{dunque} $\got{St_0(A(V))=GL(V)}$.
 \Got{Pertanto} $$\got{\forall v\in V\quad St_v(A(V))\cong St_0(A(V))=GL(V)}.$$
 
 
 
 \Got{Informalmente si può dire che in} $\got{V}$ \Got{c'è un punto privilegiato, l'origine, mentre in} $\got{A(V)}$
 \Got{tutti i punti sono equivalenti.
 Rileggendo la dimostrazione si può notare che ad ogni coppia di vettori} $\got{v,w\in V}$ \Got{si è associata una traslazione}
 $\got{\tau_{v-w}}$, \Got{e le traslazioni costituiscono un gruppo isomorfo a} $\got{(V,+)}$.
 \Got{Grazie alla definizione di spazio affine si distingueranno gli elementi di} $\got{V}$ \Got{pensati come punti o come traslazioni.}
 \end{proof}
 
	\section{Spazi affini}
 
 \Got{Dato uno spazio vettoriale} $\got{V}$, \Got{un insieme non vuoto} $\got{A}$ \Got{si dice spazio affine su} $\got{V}$
 \Got{se esiste una funzione} $\got{F:A\times A\to V}$ \Got{che associa ad ogni coppia di punti} $\got{P,Q\in V}$ \Got{un vettore di} $\got{V}$,
 \Got{denotato} $\overrightarrow{\got{PQ}}$, \Got{in modo da verificare le seguenti condizioni:}
 \begin{enumerate}[label=\bf\Roman*)]
	\item $\got{\forall P\in A,\forall v\in V\;\;\exists!\; Q\in A}$ \Got{tale che} $\overrightarrow{\got{PQ}}=\got{v}$;
	\item $\got{\forall P,Q,R\in A}\quad\overrightarrow{\got{PQ}}+\overrightarrow{\got{QR}}=\overrightarrow{\got{PR}}\;\;$
		\Got{(relazione di Chasles)}.
 \end{enumerate}
 
 \begin{remark}
 \Got{Dalla relazione di Chasles discende che:} 
 \begin{enumerate}[label=\bf\alph*)]
	\item $\got{\forall P\in A}\quad\overrightarrow{\got{PP}}=\got{0}\quad$ \Got{(prendendo} $\got{P=Q=R);}$
	\item $\got{\forall P,Q\in A}\quad\overrightarrow{\got{PQ}}=-\overrightarrow{\got{QP}}\quad$ \Got{(prendendo} $\got{P=R).}$
 \end{enumerate}
 \end{remark}
 
 \begin{esempi}
	\begin{enumerate}[label=\bf\Roman*)]
 \item $\got{A=V}$,
 \begin{eqnarray*}
	\got{F:} \got{V}\times \got{V} & \to & \got{V} \\
	\got{(P,Q)} & \mapsto & \vet{PQ}\stackrel{\got{def}}{=}\got{Q-P}
 \end{eqnarray*}
 \item Data $\got{f\in Hom(V,W)}$ \Got{e dato} $\got{b\in W}$ \Got{siano} $\got{A=f^{-1}(b)}$ \Got{e} $\got{V=f^{-1}(0)=ker(f)}$.
 \Got{Allora} $\got{A}$ \Got{è uno spazio affine su} $\got{V}$ \Got{tramite}
 \begin{eqnarray*}
  \got{F:} \got{A\times A} & \to & \got{V} \\
  \got{(a_1,a_2)} & \mapsto & \vet{a_1a_2}\stackrel{\got{def}}{=}\got{a_2-a_1}
 \end{eqnarray*}
 \end{enumerate}
 \end{esempi}
 
	\subsection{Traslazioni}
 
 \Got{Dalla definizione, fissato} $\got{v\in V,\;\exists !Q\in A}$ \Got{tale che} $\got{\vet{PQ}=\got{v}}$.
 \Got{Si definisce allora traslazione ogni funzione} $\got{\tau_v:A\to A}$ \Got{tale che} $\got{\tau_v(P)=Q,}$ \Got{dove} $\vet{PQ}=\got{v}$.
 \Got{Inoltre si usa la notazione} $\got{P+v=\tau_v(P)}$.
 \Got{Con tale notazione si ha che:}
 \begin{itemize}
	\item $\vet{P(P+v)}=\got {v}$;
	\item $\got{P}+\vet{PQ}=\got{Q}$.
 \end{itemize}
 
 \begin{lemma}
 \Got{}$\got{\forall P\in A,\; \forall v_1,v_2\in V}$ \Got{si ha che} $\got{(P+v_1)+v_2=P+(v_1+v_2).}$
 \end{lemma}
 
 \begin{proof}
 \Got{Siano} $\got{P_1=P+v_1}$ \Got{e} $\got{P_2=P+v_2}$. \Got{Allora} $\got{v_1=\vet{PP_1}}$ \Got{e}
 $\got{v_2=\vet{PP_2}.}$.
 \Got{Perciò si ha che} $\got{P+(v_1+v_2)=P+(\vet{PP_1}+\vet{PP_2})=P+\vet{PP_2}=P_2}$.
 \end{proof}
 
	\subsection{Combinazioni affini}
 
 \Got{Fissato} $\got{P\in A}$, \Got{si definisce}
 \begin{eqnarray*}
  \got{F_P:} \got{A} & \to & \got{V} \\
  \got{Q} & \mapsto & \vet{PQ}
 \end{eqnarray*}
 
 \Got{Dagli assiomi segue che} $\got{F_P}$ \Got{è biunivoca e che} $\got{F_P(P)=\vet{PP}=0}$. \Got{Dunque} $\got{F_P}$ \Got{trasforma}
 $\got{P}$ \Got{nell'origine di} $\got{V}$.
 \Got{Dunque si vorrebbe trovare una definizione di ``combinazione affine'' di punti di} $\got{A}$ \Got{che corrisponda alla combinazione
 lineare di vettori. Siano} $\got{P_1,P_2,\ldots,P_k\in A}$. \Got{Per ogni} $\got{P\in A\; F_P:}$ \Got{trasforma}
 $\got{P_i}$ \Got{in} $\vet{PP_i}$.
 \Got{Dati} $\got{t_1,\ldots,t_k\in \K}$ \Got{esiste} $\got{\sum_{i=1}^kt_i\vet{PP_i}\in V}$, \Got{e dunque anche}
 $\got{F_P^{-1}(\sum_{i=1}^kt_i\vet{PP_i})\in A}$.
 \Got{Affinché il risultato sia indipendente da} $\got{P,}$ \Got{deve valere che}
 $$\got{\forall P,Q\in A\quad P+\sum_{i=1}^kt_i\vet{PP_i}=Q+\sum_{i=1}^kt_i\vet{QP_i},}$$
 \Got{il che accade se}
 $$\got{P+\sum_{i=1}^kt_i\vet{PP_i}=P+\sum_{i=1}^kt_i(\vet{PQ}+\vet{QP_i})=P+\left(\sum_{i=1}^kt_i\right)\vet{PQ}+\sum_{i=1}^kt_i\vet{QP_i}}$$
 \Got{è uguale a}
 $$\got{Q+\sum_{i=1}^kt_i\vet{QP_i}=P+\vet{PQ}+\sum_{i=1}^kt_i\vet{QP_i},}$$
 \Got{ossia se e solo se} $\got{\sum_{i=1}^kt_i=1}$.
 \Got{Da ciò discende la seguente definizione.}
 
 \begin{definition}
 \Got{Dati} $\got{P_1,\ldots,P_k\in A}$ \Got{e} $\got{t_1,\ldots,t_k\in \K}$ \Got{con} $\got{\sum_{i=1}^kt_i=1}$,
 \Got{si chiama combinazione affine di} $\got{P_1,\ldots,P_k}$, \Got{dato un qualsiasi} $\got{P\in A}$, 
 $\got{F_P^{-1}(\sum_{i=1}^kt_i\vet{PP_i})}$.\\
 \Got{Si definisce inoltre} $\got{comb_a(P_1,\ldots,P_n)=\{\text{combinazioni affini di }\K\}}.$
 \end{definition}
 
 \begin{esempio}
 \Got{Se} $\got{A=\K^n}$ \Got{e} $\got{P_1,P_2}$ \Got{sono punti distinti, allora}
 \begin{eqnarray*}
	\got{comb_a(P_1,P_2)=\{t_1P_1+t_2P_2\;|\;t_1+t_2=1\}=\{tP_1+(1-t)P_2\;|\; t\in \K\}=}\\
	\got{=\{P_1+(1-t)(P_2-P_1)\;|\; t\in \K\},}
 \end{eqnarray*}
 \end{esempio}

 \Got{che è la retta passante per} $\got{P_1}$ \Got{e} $\got{P_2.}$ \Got{Analogamente la combinazione affine di tre punti non allineati
 è il piano passante per essi.}
 
	\subsection{Sottospazi affini}
 
 \Got{Un sottoinsieme} $\got{H}$ \Got{di} $\got{A}$ \Got{si dice sottospazio affine se è chiuso per combinazioni affini.}
 
 \vspace{0.5cm}
 
 \begin{remark}
 \Got{L'intersezione di sottospazi affini è un sottospazio affine.}
 \end{remark}
 
 \vspace{0.5cm}
 
 \begin{esempio}
 \Got{Sia} $\got{P\in A}$, \Got{e sia} $\got{W}$ \Got{un sottospazio di} $\got{V.}$ \Got{Allora}
 $$\got{F^{-1}_{P_0}(W)=\{P\in A\;|\;}\vet{P_0P}\got{\in W\}=\{P\in V\;|\;P=P_0+w, w\in W\}}.$$
 \Got{Dunque} $\got{F^{-1}_{P_0}(P_0+W)=W}$.
 \end{esempio}
 
 \vspace{0.5cm}
 
 \begin{remark}
 \Got{}$\got{P_0+W=\{\tau_w(P_0)\;|\;w\in W\}}$.
 \end{remark}
 
 \vspace{0.5cm}
 
 \begin{proposition}
 \Got{}$\got{P_0+W}$ \Got{è chiuso per combinazioni affini.}
 \end{proposition}
 
 \begin{proof}
 \Got{Siano} $\got{P_0+w_1,\dots,P_0+w_n\in P_0+W}$ \Got{e} $\got{t_1+\ldots+t_n=1}$. \Got{Allora}
 
 \[
	\got{\sum_{i=1}^nt_i(P_0+w_i)=P_0+\sum_{i=1}^nt_i\vet{P_0(P_0+w_i)}=P_0+\sum_{i=1}^nt_iw_i\in P_0+W}.
 \]
 \end{proof}
 
 \subsection{Giacitura}
 
 \begin{proposition}
 \Got{Sia} $\got{H}$ \Got{un sottospazio affine di} $\got{A}$. \Got{Allora esiste un unico sottospazio vettoriale}
 $\got{W_H}$ \Got{di} $\got{V}$, \Got{ detto giacitura di} $\got{H}$, \Got{tale che} $\got{\forall P\in H\quad H=P_0+W_H}$.
 \end{proposition}
 
 \begin{proof}
 \Got{Dato} $\got{P_0\in H}$, \Got{si cerca} $\got{W_H}$ \Got{tale che verifichi la tesi.\\
 Poiché} $\got{P_0+W_H=F^{-1}_{P_0}(W_H)}$, \Got{necessariamente}
 $$
	\got{W_H=F_{P_0}(H)=\{w\in V\;|\; P_0+w\in H\}}.
 $$
 \Got{Bisogna verificare che} $\got{W_H}$ \Got{sia un sottospazio di} $\got{V.}$
 \Got{Dati} $\got{w_1,w_2\in W_H}$ \Got{e} $\got{\alpha_1,\alpha_2\in \K,}$ \Got{si ha che} $\got{P_0+w_1\in H}$ \Got{e che}
 $\got{P_0+w_2\in H}$. \Got{Ma allora}
 \begin{eqnarray*}
	\got{P_0+\alpha_1w_1+\alpha_2w_2=P_0+\alpha_1\vet{P_0(P_0+w_1)}+\alpha_2\vet{P_0(P_0+w_2)}=}\\
	\got{=P_0+(1-\alpha_1-\alpha_2)\vet{P_0P_0}+\alpha_1\vet{P_0(P_0+w_1)}+\alpha_2\vet{P_0(P_0+w_2)}=}\\
	\got{=F^{-1}_{P_0}((1-\alpha_1-\alpha_2)\vet{P_0P_0}+\alpha_1\vet{P_0(P_0+w_1)}+\alpha_2\vet{P_0(P_0+w_2)}\in H}
 \end{eqnarray*}

 \Got{in quanto combinazione affine di} $\got{P_0,P_0+w_1}$ \Got{e} $\got{P_0+w_2\in W_H}$.
 \Got{Quanto all'unicità, bisogna verificare che il ragionamento fatto non dipenda dall'elemento di} $\got{H}$ \Got{scelto, ossia che dati}
 $\got{P_1,P_2\in H}$ \Got{valga che} $\got{F_{P_1}(H)=F_{P_2}(H)}$.
 \Got{Ma} $\got{F_{P_1}(H)=\{\vet{P_1Q}\;|\; Q\in H\},}$ \Got{e se} $\got{\vet{P_1Q}\in F_{P_1}(H)}$ \Got{si ha che}
 $$\got{\vet{P_1Q}=\vet{P_1P_2}+\vet{P_2Q}=-\vet{P_2P_1}+\vet{P_2Q}\in F_{P_2}(H),}$$
 \Got{pertanto} $\got{F_{P_1}(H)\in F_{P_2}(H)}$, \Got{ed analogamente si dimostra l'inclusione inversa.}
 \end{proof}
 
 \begin{definition}
 \Got{Se} $\got{H\subseteq A}$ \Got{è sottospazio affine, si pone} $\got{dim\;H=dim\;W_{H}}$.
 \begin{itemize}
	\item $\got{H}$ \Got{è detto retta se} $\got{dim\;H = 1}$,
	\item $\got{H}$ \Got{è detto piano se} $\got{dim\;H = 2}$,
	\item $\got{H}$ \Got{è detto iperpiano se} $\got{dim\;H = dim\;A-1}$.
 \end{itemize}
 \end{definition}
 
 \begin{remark}
 \Got{Se} $\got{H, L \subseteq A \text{ tali che } H\cap L \neq \emptyset}$ \Got{sono sottospazi affini, allora}\\
 $\got{W_{H\cap L}=W_H\cap W_L}$.
 \end{remark}
 
 \begin{definition}
 \Got{Due sottospazi affini si dicono:}
 \begin{description}
	\item[\textbullet\ incidenti] \Got{se} $\got{H\cap L \neq 0}$,
	\item[\textbullet\ paralleli] \Got{se} $\got{W_H\subseteq W_L\ \vee\ W_L\subseteq W_H}$.
 \end{description}
 \end{definition}
 
 \begin{remark}
 \Got{il parallelismo in generale non è una relazione transitiva, quindi neanche d'equivalenza.}
 \end{remark}

 \begin{proposition}
 \Got{Dati} $\got{P_0,\dots,P_k}$,
 \[
	\got{comb_a(P_0,\dots,P_k)= P_0+Span}(\vet{P_0P_1},\dots,\vet{P_0P_k}).
 \]
 \end{proposition}
 
 \begin{proof}
 \Got{Per definizione di combinazione affine vale} $\got{\subseteq}$. \Got{Per il} $\got{\supseteq}$ \Got{si ha}
 \[
	\got{P_0+t_1\vet{P_0P_1}+\dots+t_k\vet{P_OP_k} 
		=P_0+\left(1-\sum_{i=1}^{k}t_i\right)+t_1\vet{P_0P_1}+\dots+t_k\vet{P_OP_k}},
 \]
 \Got{che sta in} $\got{comb_a(P_0,\dots,P_k)}$, $\got{\forall t_1,\dots,t_k\in \K}.$
 \end{proof}
 
 \Got{Più in generale, ponendo} $\got{comb_a(X)}$ \Got{l'insieme di combinazioni affini di tutti i possibili sottoinsiemi finiti di} $\got{X}$
 \Got{si ottiene l'analogo della proposizione appena dimostrata.}
 
 \begin{notazione}
	\Got{poniamo} $\got{H+L = comb_a(H\cup L)}$.
 \end{notazione}
 
 \begin{proposition}[Giacitura di $\got{H+L}$] \Got{Se} $\got{H,L\subseteq A}$ \Got{sono sottospazi affini, allora}
 $\got{\forall P\in H}$ \Got{e} $\got{\forall Q \in L}$ \Got{si ha che}
 \[
	\got{W_{H+L}=W_H+W_L+Span(\vet{PQ})}.
 \]
 \end{proposition}

 \begin{proof}
 \Got{} $\got{H\subseteq H+L \Rightarrow W_H \subseteq W_{H+L}}$, $\got{W_L \subseteq W_{H+L}}$ \Got{; inoltre si ha che}\\
 $\got{comb_a(P,Q)\subseteq H+L \Rightarrow W_{\vet{PQ}}\subseteq W_{H+L}}$; \Got{quindi vale l'inclusione}
 \[
	\got{W_{H+L}\supseteq W_H+W_L+Span(\vet{PQ})}.
 \]
 \Got{Sia ora} $\got{S=P+W_H+W_L+Span(\vet{PQ})}$. \Got{Si ha che} $\got{S\supseteq P+W_H = H}$; \Got{e, poiché} $\got{Q= P+ \vet{PQ}\in S}$,
 $\got{L=Q+W_L \subseteq Q+W_H+W_L+Span(\vet{PQ})=S}$. \Got{Ciò significa che} $\got{S\supseteq H+L}$ \Got{per definizione di combinazione
 affine.}
 \end{proof}
 
 \begin{lemma}
 \Got{} $\got{H,L\subseteq A}$ \Got{sottospazi affini, allora vale che}
 \[
	\got{H\cap L = \emptyset\quad \Leftrightarrow\quad \forall P\in H, Q \in L,\ \vet{PQ}\notin W_H+W_L.}
 \]
 \end{lemma}
 
 \begin{proof}
 \Got{Se per assurdo esistessero} $\got{P\in H \text{e} Q\in L \text{ tali che } \vet{PQ}=w_1+w_2}$ \Got{in modo che}\\
 $\got{w_1\in W_H, w_2\in W_L}$, \Got{si avrebbe che} $\got{H \ni P+w_1 = P+(\vet{PQ}-w_2) = Q - w_2 \in L.}$
 
 \Got{Viceversa, se} $\got{\exists R \in H \cap L}$, \Got{presi} $\got{P\in H, Q \in L}$ \Got{si ha che} $\got{H=P+W_H,L=Q+W_L}$ \Got{quindi}
 $\got{\vet{PQ}=\vet{PR}+\vet{RQ}\in W_H+W_L}$.
 \end{proof}
 
 \begin{proposition}[Formula di Gra\ss mann affine]
\Got{Dati} $\got{H,L\subseteq A}$ \Got{sottospazi affini si ha che}
 \begin{enumerate}[label=\bf\Roman*)]
	\item $\got{H\cap L \neq \emptyset, \ dim\;(H+L)=dim\;H+dim\;L-dim\;(H\cap L)+1}$;\label{GrassAffnote}
	\item $\got{H\cap L = \emptyset, \ dim\;{H+L}=dim\;H+dim\;L+1}$.\label{GrassAffemp}
 \end{enumerate}
 \end{proposition}
 
 \begin{proof}
 \Got{} $\got{dim\;(H+L)= dim\;W_{H+L}}$, $\got{W_{H+L}=W_H+W_L+Span(\vet{PQ})},$ \Got{quindi si ha che:}
 \begin{description}
	\item[\ref{GrassAffnote}] \Got{per il lemma otteniamo che} $\got{\vet{PQ}\in W_H+W_L}$ \Got{cioè} $\got{W_{H+L}=W_L+W_H}$
		\Got{da cui la tesi discende dalla formula di Gra\ss mann vettoriale;}
	\item[\ref{GrassAffemp}] \Got{in questo caso} $\got{\vet{PQ}\notin W_H+W_L}$ \Got{allora} $\got{dim\;W_{H+L}=dim\;(W_L+W_H)+1}$
		\Got{e la tesi si ha ancora per la formula di Gra\ss mann vettoriale.}
 \end{description}
 \end{proof}

 \begin{definition}
 \Got{} $\got{H,L}$ \Got{si dicono sghembi se } $\got{H\cap L = \emptyset}$ \Got{e} $\got{W_H \cap W_L = \{0\}}$.
 \end{definition}
 
	\subsection{Riferimenti affini}
 
 \begin{definition}
 \Got{Dati i punti} $\got{P_0,\dots, P_k\in A,}$ \Got{essi si dicono affinemente indipendenti se si ha}
 $\got{dim\;comb_a(P_0,\dots,P_k)=k}$.
 \end{definition}
 
 \begin{definition}
 \Got{Sia} $\got{n=dim\;A}$. \Got{Ogni} $\got{n+1}$\Got{-upla di punti} $\got{\{P_0,\dots, P_n\}}$
 \Got{che siano affinemente indipendenti.}
 \end{definition}
 
 \begin{remark}[passaggio da un riferimento affine ad base di vettori]
 \Got{Se} $\got{\{P_0,\dots, P_n\}}$ \Got{è un riferimento affine, allora}
 $\got{\{\vet{P_0P_1},\dots,\vet{P_0P_n}\}}$ \Got{è una base di} $\got{V}$. \Got{Viceversa, se} $\got{\{v_1,\dots,v_n\}}$ \Got{è una base di}
 $\got{V}$ \Got{e} $\got{P\in A}$ \Got{, allora} $\got{\{P,P+v_1,\dots,P+v_n\}}$ \Got{è un riferimento affine. In particolare se}
 $\got{A=V=\K^n}$, $\got{\{0,e_1,\dots,e_n\}}$ \Got{è detto riferimento affine standard.}
 \end{remark}
 
 \begin{proposition}
 \Got{Se} $\got{R=\{P_0,\dots,P_k\}}$ \Got{è un riferimento affine di} $\got{A}$, \Got{allora}
 \[
	\got{\forall P\in A, \exists ! a_0,\dots,a_n\in \K :\ P=a_0P_0+\dots+a_nP_n.}
 \]
 \Got{In particolare, la} $\got{n+1}$\Got{-upla} $\got{a_0,\dots,a_n}$ \Got{sono detti coefficienti affini di} $\got{P}$
 \Got{rispetto a} $\got{R}$.
 \end{proposition}
 
 \begin{proof}
 \Got{Per ipotesi,} $\got{A=comb_a(P_0,\dots,P_n)}$ \Got{quindi esistono sicuramente dei coefficienti per ogni}
 $\got{P\in A}$. \Got{Per mostrare che sono unici, basta spostarsi nello spazio vettoriale}
 $\got{V}$ \Got{con il procedimento mostrato nella nota precedente.}
 \end{proof}

	\section{Trasformazioni affini}
	
 \Got{Siano} $\got{A}$ \Got{uno spazio affine su} $\got{V}$ \Got{e} $\got{B}$ \Got{uno spazio affine su} $\got{W,}$ \Got{dove} $\got{V}$ \Got{e}
 $\got{W}$ \Got{sono spazi vettoriali sul campo} $\got{\K.}$
 
 \begin{definition}
 \Got{Si dice trasformazione affine una funzione} $\got{f:A\to B}$ \Got{che conserva le combinazioni affini.}
 \end{definition}
 
 \begin{definition}
 \Got{Una trasformazione affine biunivoca si dice isomorfismo affine.}
 \end{definition}
 
 \begin{definition}
 \Got{Un isomorfismo affine da uno spazio affine in sé si dice affinità. Si definisce inoltre}
 $\got{\Aff(A)=\{f:A\to A\;|\; f\; \text{ è un'affinità}\}.}$
 \end{definition}
 
 \begin{esempi}
 \begin{enumerate}[label=\bf\Roman*)]
  \item \Got{Le traslazioni sono affinità.}
  \item \Got{Se} $\got{R}$ \Got{è un riferimento affine, la funzione}
   \begin{eqnarray*}
  \got{[\quad]_R:A} & \to & \got{\K^n} \\
  \got{P} & \mapsto & \got{[P]_R}
 \end{eqnarray*}
  \Got{(dove} $\got{[P]_R}$ \Got{sono le coordinate affini di} $\got{P}$ \Got{rispetto a} $\got{R)}$ \Got{è un isomorfismo affine;
 esso manda i punti di} $\got{R}$ \Got{in} $\got{0,e_1,\ldots,e_n.}$
 \end{enumerate}
 \end{esempi}
 
 \begin{proposition}
 \Got{Se} $\got{V}$ \Got{è uno spazio vettoriale,} $\got{f:V\to V}$ \Got{è un'affinità e} $\got{f(0)=0,}$ \Got{allora}
 $\got{f}$ \Got{è lineare.}
 \end{proposition}
 \begin{proof}
 \Got{}$\got{\forall v_1,v_2\in V\quad\forall t_1,t_2\in \K}$ \Got{vale che:}
 
 \begin{eqnarray*}
	& \got{f(t_1v_1+t_2v_2)=f((1-t_1-t_2)\cdotp 0+t_1v_1+t_2v_2=}\\
	& \got{=(1-t_1-t_2)f(0)+t_1f(v_1)+t_2f(v_2)=t_1f(v_1)+t_2f(v_2).}
 \end{eqnarray*}
 \end{proof}
 
 \begin{proposition}
 \Got{Se} $\got{f\in \Aff(v)}$ \Got{allora esistono e sono unici} $\got{v\in V}$ \Got{e} $\got{g\in GL(V)}$ \Got{tali che}
 $\got{f=\tau_v}\circ\got{ g.}$
 \end{proposition}
 
 \begin{proof}
 \Got{Sia} $\got{v=f(0).}$ \Got{Allora} $\got{g=\tau_{-v}}\circ\got{ f\in \Aff(V)}$ \Got{e} $\got{g(0)=0,}$
 \Got{quindi per la proposizione precedente} $\got{g\in GL(V).}$
 \Got{Da tale proposizione segue che} 
 $$
	\got{\Aff(V)=\{\tau_v}\circ\got{ f\;|\;v\in V,f\in GL(V)\}.}
 $$
 \Got{Di conseguenza} $\got{\Aff(V)}$ \Got{è il gruppo generato da} $\got{T(V)}$ \Got{e da} $\got{GL(V).}$
 \Got{In particolare} 
 $$
	\got{\Aff(\K^n)=\{X\mapsto AX+B \; |\; A\in GL(n,\K),B\in \K^n\}.}
 $$
 \end{proof}
 
\begin{proposition}
\Got{Dato uno spazio} $\got{A}$ \Got{affine su} $\got{V}$ \Got{e dati due riferimenti affini} $\got{L=(P_0,P_1,\ldots,P_n)}$
 \Got{e} $\got{M=(Q_0,Q_1,\ldots,Q_n)}$ \Got{esiste un'unica affinità} $\got{f\in \Aff(A)}$ \Got{tale che}
 $\got{\forall i=0,\ldots,n}$\\
 $\got{ f(P_i)=Q_i.}$
 \end{proposition}
 
 \begin{proof}
 \Got{Consideriamo le trasformazioni invertibili} $\got{F_{P_0}:A\rightarrow V}$ \Got{e} $\got{F_{Q_0}:A\rightarrow V}$,
 \Got{allora si ha che} $\got{F_{P_0}(L)=(0,\vet{P_0P_1},\dots,\vet{P_0P_n})=(0,B)}$ \Got{dove} $\got{B\in V}$ \Got{è un vettore}
 \Got{di dimensione} $\got{n}$; \Got{similmente} $\got{F_{Q_0}(M)=(0,C).}$
 
 \Got{Ora,} $\got{B,C}$, \Got{per motivi dimensionali, sono basi dello spazio} $\got{V}$, \Got{pertanto esiste un unico endomorfismo lineare}
 $\got{g}$ \Got{che mappa} $\got{B}$ \Got{in} $\got{C}$. \Got{Infine, sia} $\got{\tau:V\rightarrow V}$ \Got{la traslazione del vettore}
 $\got{\vet{P_0Q_0}}$. \Got{Mostriamo che la mappa} $\got{f:A\rightarrow A}$ \Got{definita da}
 \[
	\got{f = F_{P_0}^{-1} \circ \tau \circ g \circ F_{P_0}}
 \]
 \Got{soddisfa le condizioni richieste:}
 \begin{eqnarray*}
	&&\got{f(P_0)=F_{P_0}^{-1} \circ \tau \circ g (0) = F_{P_0}^{-1} \circ \tau (0) = F_{P_0}^{-1} (\vet{P_0Q_0}) = Q_0}\\
	&&\got{f(P_i)=F_{P_0}^{-1} \circ \tau \circ g (\vet{P_0P_i}) = F_{P_0}^{-1} \circ \tau (\vet{Q_0Q_i}) = F_{P_0}^{-1} (\vet{P_0Q_i}) = Q_i}
 \end{eqnarray*}
 $\got{\forall i=1,\dots,n}.$
 \Got{Mostriamo che conserva anche le combinazioni affini, in particolare basta mostrare che conserva quelle di punti in} $\got{L}$:
 \begin{eqnarray*}
	\got{f\left(\sum_{i=0}^{n}t_iP_i\right)}&=&\got{F_{P_0}^{-1} \circ \tau \circ g \left(\sum_{i=0}^{n}t_i\vet{P_0P_i}\right)}\\
	&=&\got{F_{P_0}^{-1} \circ \tau \left( \sum_{i=0}^{n}t_i\vet{Q_0Q_i}\right) }\\
	&=&\got{F_{P_0}^{-1} \left(\sum_{i=0}^{n}t_i\vet{P_0Q_i}\right)}\\
	&=&\got{\sum_{i=0}^{n}t_iQ_i}.
 \end{eqnarray*}
 
 \Got{D'altra parte, si verifica facilmente che se esiste} $\got{f}$ \Got{siffatta, allora l'applicazione} $\got{j:V\rightarrow V}$ 
 \Got{definita}
 \[
	\got{j = \tau^{-1} \circ F_{P_0} \circ f \circ F_{P_0}}
 \]
 \Got{manda } $\got{\vet{P_0P_i}}$ \Got{in} $\got{\vet{Q_0Q_i}\ \forall i=1,\dots,n}$
 \Got{e che è lineare (la verifica è del tutto analoga ai passaggi appena descritti). Da ciò segue quindi che } $\got{g=j}$ 
 \Got{cioè l'unicità.}
 \end{proof}
 
	\subsection{Gruppo delle affinità}
 
 \begin{definition}[gruppo delle affinità]
 \Got{consideriamo il seguente insieme}
 \[
	\got{\Aff(\K^n)=\{x\mapsto Mx+N : M\in GL{\K^n}, N\in \K^n\}}
 \]
 \Got{e l'operazione di composizione} $\circ$: \Got{la coppia} $\got{(\Aff(\K^n),\circ))}$ \Got{si dice gruppo delle affinità di} $\got{\K^n}$.
 \end{definition}
 
 \begin{remark}
 \Got{la mappa} $\phi:\K^n\rightarrow \K^{n+1}$, $x\mapsto\left( \begin{matrix}
                                                              x\\
                                                              \hline
                                                              1
                                                             \end{matrix} \right)
$ \Got{è un isomorfismo affine tra} $\got{\K^n}$ \Got{e}\\
$\got{H=\{x\in \K^{n+1} : x_{n+1}=0\}}$ \Got{il quale è sottospazio affine di } $\got{\K^{n+1}}$.
 \end{remark}
 
 \Got{Vogliamo ora far vedere che questo isomorfismo è più profondo di quanto non appaia a prima vista. In particolare cercheremo di studiare}
 \Got{la relazione tra le trasformazioni affini in} $\K^n$ \Got{e le trasformazioni lineari in} $\K^{n+1}$ \Got{arrivando ad identificare}
 $\got{\Aff(\K^n)}$ \Got{con un sottogruppo di} $\got{GL(n,\K)}$.
 
 \begin{definition}
 \Got{} $\got{G(H)=\{g\in GL(n+1,\K):g(H)=H\}}$.
 \end{definition}
 
 \begin{remark}
 \Got{\'E immediato verificare che } $\got{G(H)}$ \Got{è un sottogruppo di } $\got{GL(n+1,\K)}$,
 \Got{dato che composizione ed inverse di mappe che conservano} $\got{H}$ \Got{a loro volta lo conservano}.
 \end{remark}
 
 \begin{remark}
 \Got{Se scomponiamo } $\got{g\in G(H)}$ \Got{in blocchi in questo modo:}
 \[
	\got{g} = \left(
		\begin{array}{cc}
		\got{M} & \got{N} \\ 
		\got{^tp} & \got{q}
		\end{array}
	\right)
 \]
 \Got{ dove} $\got{M\in M(n,\K);\  N,p\in \K^n,\ q \in \K}$, \Got{e se vogliamo che } $\got{g}$ \Got{fissi} $\got{H}$
 \Got{allora si deve avere che la} $\got{n+1}$ \Got{-esima componente di} $\got{g\left( \begin{matrix}
                                                              x\\
                                                              \hline
                                                              1
                                                             \end{matrix} \right)}$ \Got{sia ancora} $\got{1}$,
 \Got{cioè che } $\got{\forall x\in \K^n\  ^tpx+q=1}$, \Got{cioè che} $\got{p=0, q=1}$.
 \end{remark}
 
 \begin{proposition}
 \Got{sia } $\Phi:\got{\Aff(\K^n)\rightarrow G(H)}$ \Got{definito da}
 \[
	\Phi\got{(x\mapsto Mx+N)=}\left(
		\begin{array}{cc}
		\got{M} & \got{N} \\ 
		\got{0} & \got{1}
		\end{array}
	\right)
 \]
 \Got{è un isomorfismo di gruppi.}
 \end{proposition}
 
 \begin{proof}
 \Got{Per prima cosa è immediato vedere che} $\Phi$ \Got{ è iniettivo dato che il nucleo contiene solo l'identità.}
 \Got{La surgettività ci è fornita dalla precedente osservazione. Rimane da far vedere che questo operatore commuta con la composizione.}
 \Got{Date due affinità} $\got{a:x\mapsto Mx+N}$ \Got{e} $\got{a':x\mapsto M'x+N',}$ \Got{si ha che}
 \[
	\got{a\circ a':x\mapsto MM'x+ MN'+N}
 \]
 \Got{quindi}
 \[
	\Phi\got{(a)}\circ\Phi\got{(a')}=\left(
		\begin{array}{cc}
		\got{MM'} & \got{MN'+N} \\ 
		\got{0} & \got{1}
		\end{array}
	\right)=\Phi\got{(a\circ a')}.
 \]
 \end{proof}
 
	\section{Equivalenza affine}

 \begin{definition}
 \Got{Se} $\got{G}$ \Got{è un gruppo di trasformazioni di} $\got{\K^n}$ \Got{e} $\got{F_1}$ \Got{e} $\got{F_2}$
 \Got{sono sottoinsiemi di} $\got{\K^n}$ \Got{allora} $\got{F_1}$ \Got{e} $\got{F_2}$ \Got{si dicono equivalenti per} $\got{G}$ \Got{se}
 $\got{\exists g\in G}$ \Got{tale che} $\got{g(F_1)=F_2.}$
 \end{definition}
 
 \Got{In particolare:}
 
 \begin{definition}
 \Got{}$\got{F_1,F_2\in \K^n}$ \Got{si dicono affinemente equivalenti se} $\got{\exists g\in \Aff(\K^n)}$ \Got{tale che}\\
 $\got{g(F_1)=F_2.}$
 \end{definition}
 
 \Got{Analogamente} $\got{F_1,F_2\in \K^n}$ \Got{si dicono metricamente equivalenti se} $\got{\exists g\in \Isom(\K^n)}$ \Got{tale che}\\
 $\got{g(F_1)=F_2.}$
 
 \begin{esempio}
 \Got{Se} $\got{F_1=\{P_0,\ldots,P_n\}}$ \Got{e} $\got{F_2=\{Q_0,\ldots,Q_n\},}$ \Got{sono insiemi di punti affinemente
 indipendenti, per quanto visto} $\got{\exists g\in \Aff(\K^n)}$ \Got{tale che} $\got{g_(F_1)=F_2,}$ \Got{dunque essi sono affinemente
 equivalenti. In particolare gli iperpiani affini sono affinemente equivalenti tra di loro.}
 \end{esempio}
 
 \Got{Dunque sorgono alcune domande.}
 
 \begin{problemaI}
  \Got{Data una famiglia} $\got{F}$ \Got{di insiemi di} $\got{\K^n}$ \Got{stabilire quando due suoi elementi sono affinemente
 equivalenti.}
 \end{problemaI}
 
 \begin{problemaI}\Got{Data una famiglia} $\got{F}$ \Got{di insiemi di} $\got{\K^n}$ \Got{trovare un insieme di ``modelli'' (sive forme
 canoniche)} $\got{\{E_i\}_{i\in I}\subseteq F}$ \Got{tali che:}
 \[
	\left\{
		\begin{array}{l}
			\got{\forall i\ne j\quad E_i}\text{ ed } E_j\text{non sono affinemente equivalenti}\\
			\got{\forall S\in F\quad\exists E_i}\quad \text{affinemente equivalente a } S
		\end{array}
	\right.
 \]
 \end{problemaI}
 
 \begin{ossI}
\Got{Se} $\got{F}$ \Got{è la famiglia di iperpiani affini, c'e una sola classe di equivalenza.}
\end{ossI}
 
 \begin{ossII}
 \Got{Un iperpiano è la soluzione di un'equazione lineare (Un polinomio di primo grado in più variabili.}
 \end{ossII}
 
  \vspace{0.5cm}
 
 \begin{notazione}
 \Got{Se} $\got{g\in \K[ x_1,\ldots,x_n]}$ \Got{si definisce} $\got{V(g)=\{x\in \K^n\;|\;g(x)=0\}.}$
 \end{notazione}
 
 \begin{remark}
 \Got{La funzione}
 \[
	\begin{array}{rcl}
		\got{\K[ x_1,\ldots,x_n]} & \to & \got{\K^n} \\
		\got{g} & \mapsto & \got{V(g)}
	\end{array}
 \]
 \Got{non è iniettiva:}
	\begin{itemize}
		\item $\got{\forall\alpha\in \K\setminus\{0\}\quad V(\alpha g)=V(g)}$
		\item $\got{\forall n\in N\setminus\{0\}\quad V(g^n)=V(g)}$
		\item \Got{se} $\got{\K=C}$ \Got{e se} $\got{g_1}$ \Got{e} $\got{g_2}$ \Got{non contengono fattori multipli allora} $\got{V(g_1)=V(g_2)}$
		\item \Got{se} $\got{\K=R}$ \Got{quest'ultima proprietà non vale: ad esempio}
			$$\got{\forall c>0\quad V(x^2+y^2+c)=\emptyset}$$
	\end{itemize}
 \end{remark}

 \begin{definition}
 \Got{}$\got{g_1}$ \Got{e} $\got{g_2\in \K[x_1,\ldots,x_n]}$ \Got{si dicono proporzionali, e si scrive}
 $\got{g_1\sim g_2,}$\\
 \Got{se} $\got{\exists\alpha\in \K\setminus\{0\}}$ \Got{tale che} $\got{g_1=\alpha g_2.}$
 \end{definition}

  \section{Ipersuperfici affini}
 
 \Got{Si chiama ipersuperficie affine di grado} $\got{n}$ \Got{ogni classe di proposizionalità di polinomî in} $\got{\K^n.}$
 
 \Got{Se} $\got{I=[g]}$ \Got{è un'ipersuperficie, allora} $\got{g(0)=0}$ \Got{è detta equazione dell'ipersuperficie.}
 
 $\got{V(g)\in \K^n}$ \Got{è detto supporto di} $\got{I.}$
 
 \begin{remark}
 \begin{itemize}
  \item \Got{Se} $\got{g_1\sim g_2}$ \Got{allora} $\got{g_1}$ \Got{e} $\got{g_2,}$ \Got{hanno lo stesso grado. Dunque si può definire il grado
  di} $\got{I=[g]}$ \Got{come il grado dei polinomî contenuti in} $\got{I.}$
  \item \Got{Se} $\got{n=2,}$ \Got{le ipersuperfici affini vengono dette curve affini.}
  \item \Got{Se} $\got{n=3,}$ \Got{le ipersuperfici affini vengono dette superfici affini.}
  \item \Got{Gli iperpiani sono ipersuperfici di grado} $\got{1.}$
  \item \Got{Le ipersuperfici di secondo grado vengono dette quadriche (o coniche se} $\got{n=2).}$
 \end{itemize}
 \end{remark}
 
 \begin{remark}
 \Got{Un ipersuperficie determina il supporto, ma come già visto non vale il viceversa. Tuttavia conviene definire
 l'equivalenza affine a partire dalle equazioni e non dai supporti.}
 \end{remark}
 
 \begin{definition}
 \Got{Data un'ipersuperficie} $\got{I=[g]}$ \Got{e data} $\got{\psi(x)=Mx+N}$ \Got{affinità di} $\got{\K^n,}$
 \Got{si denota con} $\got{\psi^{-1}(I)}$ \Got{l'ipersuperficie di equazione} $\got{g(\psi(x))=0}$
 \end{definition}
 
\end{document}
